\documentclass[11pt,oneside]{book} % oneside for single-sided layout

%================ Dependencies ======================
% Required packages and their purposes:
% - graphicx: For including and manipulating images (emblem)
% - amsmath,amssymb,amsfonts: Mathematical symbols and equations
% - biblatex: Modern bibliography management
% - geometry: Page layout and margins
% - pgffor: For foreach loops in macros
% - hyperref: For interactive elements and PDF features
%
% Note: Make sure all these packages are installed in your LaTeX distribution.
% If you encounter any "package not found" errors, you may need to install
% the missing packages using your distribution's package manager.

%================ Packages ======================
\usepackage{graphicx}       % for including images (e.g. emblem)
\usepackage{amsmath,amssymb,amsfonts} % common math packages
\usepackage{biblatex}       % bibliography management (recommended):contentReference[oaicite:10]{index=10}
\addbibresource{references.bib} % Bibliography file
\usepackage{geometry}       % set page margins
\geometry{a4paper, margin=1in} 
\usepackage{pgffor}        % for foreach loops
\usepackage{hyperref}      % for interactive elements

%================ Metadata ======================
\newcommand{\BookTitle}{2025 SAT Summer Class}
\newcommand{\BookSubtitle}{Week 3}
\newcommand{\BookAuthor}{Jaehoon Song}
\newcommand{\BookInstitution}{Hans edu LLC (Columbia Academy)}
\newcommand{\BookEmblem}{./images/emblem.png} % Path to emblem image file
%\newcommand{\BookSubtitle}{Subtitle (optional)} % uncomment if needed





% ------------------------------------------------------------------------------
% \signField[<Printed Name>][<Width Factor>]: Generates a signature field
% ------------------------------------------------------------------------------
% DEPENDENCIES:
% - hyperref: Required for \TextField command
% - tabbing: Built-in LaTeX environment
%
% DESCRIPTION:
% This command creates a structured signature field with labeled sections for:
% - Signature
% - Printed Name (customizable label)
% - Date
% - Initials
%
% The width of the entire signature field can be adjusted using the second argument.
% The proportions remain the same while scaling.
%
% USAGE:
%   \signField[<Printed Name>][<Width Factor>]
%   - <Printed Name> (Optional): Defaults to "Printed Name"
%   - <Width Factor>: A decimal value specifying the fraction of \textwidth
%
% EXAMPLES:
%   - \shortSignField           
%   - \signField                
%   - \signField[Full Name]     → Uses "Full Name" for the name label by their usage.
% ------------------------------------------------------------------------------
\newcommand{\signField}[1][Printed Name]{
    \stepcounter{nameCounter}
    \begin{tabbing}
        \hspace{0.3\textwidth} \= \hspace{0.35\textwidth} \= \hspace{0.20\textwidth} \= \hspace{0.15\textwidth} \kill
        \makebox[0.3\textwidth]{} \> \TextField[name=name\arabic{nameCounter},width=0.35\textwidth,bordercolor=0 0 0]{} \> \TextField[name=date\arabic{nameCounter},width=0.20\textwidth,bordercolor=0 0 0]{} \> \makebox[0.15\textwidth]{} \\[-8pt]
        \underline{\hspace{0.3\textwidth}} \> \underline{\hspace{0.35\textwidth}} \> \underline{\hspace{0.20\textwidth}} \> \underline{\hspace{0.15\textwidth}} \\ 
        \textit{Signature} \> #1 \> Date \> \makebox[0.15\textwidth][r]{\textbf{\textit{Initials}}}
    \end{tabbing}
}

% ------------------------------------------------------------------------------
% \shortSignField[<Printed Name>]: Generates a compact signature field
% ------------------------------------------------------------------------------
% DEPENDENCIES:
% - hyperref: Required for \TextField command
% - tabbing: Built-in LaTeX environment
%
% DESCRIPTION:
% Creates a simplified signature field with:
% - Signature
% - Printed Name (customizable)
% - Date
%
% USAGE:
%   \shortSignField[<Printed Name>]
%   - <Printed Name> (Optional): Defaults to "Printed Name"
%
% EXAMPLES:
%   - \shortSignField           → Uses default "Printed Name" label
%   - \shortSignField[Name]     → Uses "Name" for the name label
% ------------------------------------------------------------------------------
\newcommand{\shortSignField}[1][Printed Name]{%
    \stepcounter{nameCounter}

    % Define overall width as a multiple of \textwidth
    \newlength{\overallWidth}
    \setlength{\overallWidth}{0.8\textwidth}  % Slightly reduced total width

    % Calculate proportional column widths
    \newlength{\colOne}
    \newlength{\colTwo}
    \newlength{\colThree}

    \setlength{\colOne}{0.3\overallWidth}
    \setlength{\colTwo}{0.4\overallWidth}
    \setlength{\colThree}{0.3\overallWidth}

    \begin{tabbing}
        \hspace{\colOne} \= \hspace{\colTwo} \= \hspace{\colThree} \kill
        \makebox[\colOne]{} \> 
        \TextField[name=name\arabic{nameCounter},width=\colTwo,bordercolor=0 0 0]{} \> 
        \generatedOn \\[-8pt]
        \underline{\hspace{\colOne}} \> 
        \underline{\hspace{\colTwo}} \> 
        \underline{\hspace{\colThree}} \\ 
        \textit{Signature} \> #1 \> Date
    \end{tabbing}
}

% ------------------------------------------------------------------------------
% \customtitle{<Title>}{<Subtitle>}{<Team>}{<Authors>}{<Course>}{<Institution>}{<Professor>}
% ------------------------------------------------------------------------------
% DEPENDENCIES:
% - graphicx: Required for \includegraphics command
% - hyperref: Required for PDF metadata
%
% DESCRIPTION:
% Creates a professional title page with:
% - Institution logo
% - Title and subtitle
% - Team information
% - Author list
% - Course information
% - Institution name
% - Professor name
%
% USAGE:
%   \customtitle{<Title>}{<Subtitle>}{<Team>}{<Authors>}{<Course>}{<Institution>}{<Professor>}
%
% PARAMETERS:
%   - Title: Main title of the document
%   - Subtitle: Secondary title or project description
%   - Team: Team name or identifier
%   - Authors: List of authors (use \convertarraytostring for multiple authors)
%   - Course: Course name and code
%   - Institution: Institution name
%   - Professor: Professor's name
%
% EXAMPLES:
%   \customtitle
%     {Final Team Project: Process Book}
%     {Music Trends and Insights}
%     {Team: Vizualytics}
%     {Author One, Author Two}
%     {CS-4460-B: Introduction to Information Visualization}
%     {Georgia Institute of Technology}
%     {Professor: John Doe}
% ------------------------------------------------------------------------------
\newcommand{\convertarraytostring}[2]{
  \gdef#1{}
  \foreach \x in {#2} {
    \xdef#1{#1 \x \\}
  }
}
\newcommand{\customtitle}[7]{
    \title{
        % \includegraphics[width=0.60\textwidth]{\BookEmblem}\par\vspace{1cm}
        \includegraphics[width=0.50\textwidth]{\BookEmblem} \\ % fixed Logo settings
        \vspace*{0.5in}
        \textbf{#1} \\ 
        \vspace*{0.5in}
        #2
    }
    \author{
        #3 \\ 
        \vspace*{1pt} \\
        #4 \\
        \vspace*{0.5in} \\
        \textbf{#5} \\ 
        \vspace*{1pt} \\
        #6 \\ 
        \vspace*{1pt} \\
        \textbf{#7} \\ 
    }
    \date{\today} % fixed date settings
    \maketitle
    \newpage
}

























%================= Document ======================
\begin{document}


\frontmatter

% Title Page (title, author, institution, emblem)
\convertarraytostring{\titleauthors}{\BookAuthor  (Lecturer)}% Authors
\customtitle
{\BookTitle}% Title
{\BookSubtitle}% Subtitle
{\space}% Team name (e.g., "Team: Vizualytics")
{\titleauthors}
{SAT/DSAT/SSAT}% Course name (e.g., "CS-4460-B: Introduction to ...")
{\BookInstitution}% Institution (e.g., "Georgia Institute of Technology")
{\space}% Professor name (e.g., "Yalong Yang")





% % Title Page (title, author, institution, emblem)
% \begin{titlepage}
%   \thispagestyle{empty}
%   \begin{center}
%     \includegraphics[width=0.60\textwidth]{\BookEmblem}\par\vspace{1cm}
%     {\Huge \BookTitle \par}
%     \vspace{0.5cm}
%     {\large \BookSubtitle \par}
%     \vspace{1cm}
%     {\Large \BookAuthor \par} 
%     {\normalsize Lecturer, \BookInstitution \par}
%     \vfill
%     {\normalsize \today \par} % date or edition info
%   \end{center}
% \end{titlepage}

% \cleardoublepage
% \thispagestyle{empty}










% Copyright Page
\begin{center}
  {\small \textcopyright\ \the\year\ Hans edu LLC. All rights reserved.}\\[0.5cm]
  {\small Written by \BookAuthor}\\[0.5cm]
  {\small This document is for educational use only and is not for commercial distribution.}\\[0.5cm]
  {\small No part of this work may be reproduced without permission.}\\[0.5cm]
  {\small The author disclaims any liability for errors; use at your own risk.}
\end{center}

\cleardoublepage
\thispagestyle{empty}
% Dedication (optional)
% Uncomment and fill in to include a dedication page
%\chapter*{Dedication}
%{\itshape To [whomsoever it may concern]\par}

\cleardoublepage
\thispagestyle{empty}
% Table of Contents
\tableofcontents

\cleardoublepage
% Preface (Author's Note)
\chapter*{Preface}
\addcontentsline{toc}{chapter}{Preface}
% [Write your preface or author's note here.]

\cleardoublepage
% Acknowledgements
\chapter*{Acknowledgements}
\addcontentsline{toc}{chapter}{Acknowledgements}
% [Write acknowledgements here (contributors, funding, etc.).]

\cleardoublepage
% Introduction
\chapter*{Introduction}
\addcontentsline{toc}{chapter}{Introduction}
% [Write the introduction to the book here.]

\cleardoublepage
% List of Symbols / Notation
\chapter*{List of Symbols}
\addcontentsline{toc}{chapter}{List of Symbols}
\begin{center}
  \textbf{Symbol} \quad\quad\quad\ \textbf{Meaning}\\[0.2cm]
  \begin{tabular}{ll}
    $a$ & Description of symbol $a$.\\
    $b$ & Description of symbol $b$.\\
    $\alpha,\beta$ & Common notation (e.g., angles or coefficients).\\
    % Add more symbols as needed.
  \end{tabular}
\end{center}

\mainmatter
%=============== Main Matter ===============
% Begin Arabic page numbering and numbered chapters here.

\chapter{First Chapter Title}
% [Main text begins here; include sections, definitions, examples, etc.]

\section{Example Section}
Here is some example content. You can reference equations like \(E=mc^2\) and cite sources using \verb|\cite{key}|.  Footnotes can be added with \verb|\footnote{...}|.  For example, this is a footnote.\footnote{This is a sample footnote.}

% Add chapters, sections, exercises, etc., as needed for lecture content.

%=============== Back Matter ===============
\backmatter
% Bibliography / References
\printbibliography[heading=bibintoc] % include bibliography in TOC
% Alternatively, use \bibliography{references} if using BibTeX.

\end{document}
