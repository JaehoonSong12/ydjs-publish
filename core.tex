%%%%%%%%%%%%%%%%%%%%%%%%%%%%%%%%%%%%%%%%%%%%%%%%%%%%%%%%%%%%%%%%%%%%%%%%%%%%%%
% `preamble.tex` (core.tex) is a file commonly used in LaTeX projects to define 
% custom configurations and packages for the document. It typically includes 
% settings that control the overall appearance and behavior of the document, 
% such as font size, margins, and the inclusion of external packages for 
% enhanced functionality.
%%%%%%%%%%%%%%%%%%%%%%%%%%%%%%%%%%%%%%%%%%%%%%%%%%%%%%%%%%%%%%%%%%%%%%%%%%%%%%
% Suppress overfull and underfull box warnings
\hbadness=10000
\vbadness=10000
%%%%%%%%%%%%%%%%%%%%%%%%%%%%%%%%%%%%%%%%%%%%%%%%%%%%%%%%%%%%%%%%%%%%%%%%%%%%%%
% @requires
% 1. VSCode extension: LaTeX Workshop by James Yu
% 2. Command Palette (`View` > `Command Palette`) `Preferences: Open Keyboard Shortcuts (JSON)`
% 
% [
%   {
%     "key": "ctrl+b",
%     "command": "editor.action.insertSnippet",
%     "when": "editorLangId == 'latex'",
%     "args": {
%       "snippet": "\\textbf{${TM_SELECTED_TEXT}}"
%     }
%   },
%   {
%     "key": "ctrl+i",
%     "command": "editor.action.insertSnippet",
%     "when": "editorLangId == 'latex'",
%     "args": {
%       "snippet": "\\textit{${TM_SELECTED_TEXT}}"
%     }
%   }
% ]
% 
% 
% 3. Command Palette (`Ctrl` + `Shift` + `P`) `Preferences: Open User Settings (JSON)`, global settings
% 
% "latex-workshop.latex.synctex": 1
% 
% @usage
% 1. import configuration: 
%     ```
%     \input{preamble.tex}
%     ```
% 2. bold: `Ctrl` + `B`
% 3. italic: `Ctrl` + `I`
% 4. PDF Backtracking LaTeX Code: `Ctrl` + `Click`
%%%%%%%%%%%%%%%%%%%%%%%%%%%%%%%%%%%%%%%%%%%%%%%%%%%%%%%%%%%%%%%%%%%%%%%%%%%%%%

% system - comments
\usepackage{comment}                                % \begin{comment}...\end{comment}

% layout - margins
\usepackage{geometry}
\geometry{left=36pt, right=36pt, top=72pt, bottom=72pt}     % 1 in = 72 pt

% font - colors
\usepackage{xcolor}
\definecolor{redcolor}{rgb}{1.0, 0.0, 0.0}          % {\color{redcolor}...}
\definecolor{greencolor}{rgb}{0.0, 1.0, 0.0}        % {\color{greencolor}...}
\definecolor{bluecolor}{rgb}{0.0, 0.0, 1.0}         % {\color{bluecolor}...}
\definecolor{yellowcolor}{rgb}{1.0, 1.0, 0.0}       % {\color{yellowcolor}...}
\definecolor{cyancolor}{rgb}{0.0, 1.0, 1.0}         % {\color{cyancolor}...}
\definecolor{magentacolor}{rgb}{1.0, 0.0, 1.0}      % {\color{magentacolor}...}
\definecolor{blackcolor}{rgb}{0, 0, 0}              % {\color{blackcolor}...}
\definecolor{darkgreen}{RGB}{0,128,0}

% font - size
                                                    % \tiny This is tiny text. \normalsize
                                                    % \scriptsize This is scriptsize text. \normalsize
                                                    % \footnotesize This is footnotesize text. \normalsize
                                                    % \small This is small text. \normalsize
                                                    % \normalsize This is normalsize text. \normalsize
                                                    % \large This is large text. \normalsize
                                                    % \Large This is Large text. \normalsize
                                                    % \LARGE This is LARGE text. \normalsize
                                                    % \huge This is huge text. \normalsize
                                                    % \Huge This is Huge text. \normalsize

% paragraph - indentation
\setlength{\parindent}{0pt}                         % default: 18pt





% resource - hyperlinks


% resource - hyperlinks
\usepackage{hyperref}                               % \href{URL}{URL}.

% \href{
%     https://www.kaggle.com/datasets/kyanyoga/sample-sales-data
% }{Sample Sales Data}







% This command inserts an image into the document with a customizable size, file path, and caption.
%
% Parameters:
%   #1: Size of the image relative to the text width. The value should be a number (e.g., 0.5 for half the width).
%   #2: File path of the image, relative to the current directory (e.g., images/filename.png).
%   #3: Caption text to be displayed below the image.
%
% Usage:
%   The \insertimage command automatically formats the image and places it within the document, ensuring proper
%   positioning and captioning.
%
% Example:
%   \insertimage{0.8}{images/graph.png}{This is an example caption.}
%
% The figure will be centered, and the image size will be 80% of the text width.
\usepackage{graphicx}
\newcommand{\insertimage}[3]{                           % \ref{fig:images/filename}
    \begin{figure}[h!]
        \centering
        \includegraphics[width=#1\textwidth]{#2}
        \caption{#3}
        \label{fig:#2}
    \end{figure}
    \ \newline
}

% resource - videos
\usepackage{media9}                                 % \insertvideo{width}{height}{video path}{caption}{autoplay (true/false)}{loop (true/false)}
\newcommand{\insertvideo}[6]{                           % \ref{fig:video path}
    \begin{figure}[h!]
        \centering
        \includemedia[
            width=#1\textwidth,        % Video width (relative to text width)
            height=#2\textwidth,       % Video height (relative to text width)
            activate=onclick,          % Activate the video on click
            addresource=#3,            % Video file path (e.g., video.mp4)
            flashvars={
                src=#3 % Video source
                &autoPlay=#5 % Autoplay true/false
                &loop=#6     % Loop true/false
            }
        ]{}{VPlayer.swf}
        \caption{#4}                   % Caption for the video
        \label{fig:#3}                 % Label using the video path for easy referencing
    \end{figure}
}



% paragraph - code
\usepackage{listings, lmodern}                               % \begin{lstlisting}\end{lstlisting}
\lstset{
  language=Python,
  commentstyle=\color{darkgreen},
  keywordstyle=\color{blue},
  stringstyle=\color{darkgreen},
  basicstyle=\ttfamily,
  breaklines=true,
  numberstyle=\tiny\color{gray},
  frame=single,
  literate=*{0}{{{\color{blue}0}}}1
            {1}{{{\color{blue}1}}}1
            {2}{{{\color{blue}2}}}1
            {3}{{{\color{blue}3}}}1
            {4}{{{\color{blue}4}}}1
            {5}{{{\color{blue}5}}}1
            {6}{{{\color{blue}6}}}1
            {7}{{{\color{blue}7}}}1
            {8}{{{\color{blue}8}}}1
            {9}{{{\color{blue}9}}}1
}



% Tabular Data (Tables)
\usepackage{array}
\usepackage{lipsum}
\usepackage{longtable}








% ------------------------------------------------------------------------------
% \generatedOn{<locale>}: Prints the exact date and time with regional formatting
% ------------------------------------------------------------------------------
% DESCRIPTION:
% This command prints the current date and time in a readable way, formatted based on
% the locale passed as an argument (e.g., "en-US", "ko-KR", "ja-JP").
%
% The format includes:
% - Time in 12-hour format with AM/PM (in the specified region's style)
% - Day of the week
% - Month (full name)
% - Day of the month
% - Year
% - Timezone abbreviation (depending on system locale)
%
% USAGE:
% \generatedOn{<locale>} where <locale> is one of the following:
%   - "en" for English format (12-hour with AM/PM)
%   - "ko" for Korean format (24-hour, day of the week, etc.)
%   - "ja" for Japanese format (similar to Korean style)
%
% EXAMPLE USAGE:
% \generatedOn{en} will output: "11:59 PM EST, Wednesday, January 29, 2025"
% \generatedOn{ko} will output: "2025년 1월 29일 수요일 23:59"
% \generatedOn{ja} will output: "2025年1月29日水曜日 23:59"
% ------------------------------------------------------------------------------
\usepackage{datetime2}
% ------------------------------------------------------------------------------
% Command: \generatedOn
% ------------------------------------------------------------------------------
% \generatedOn: Prints the date and time of document compilation.
% ------------------------------------------------------------------------------
% DESCRIPTION:
% This command prints the exact date and time when the LaTeX document was 
% compiled. It uses the `datetime2` package to fetch the timestamp in the ISO 
% 8601 format (YYYY-MM-DD HH:MM:SS).
%
% USAGE:
% To use this command in your document, simply invoke `\generatedOn` where you 
% want the compilation timestamp to appear.
%
% EXAMPLE:
% The document was generated on: \generatedOn
% ------------------------------------------------------------------------------

\newcommand{\generatedOn}{\DTMnow}  % Automatically generates the current date and time









%%%%%%%%%%%%%%%%%%%%%%%%%%%%%%%%%%%%%%%%%%%%%%%%%%%%%%%%%%
%%%%%%%%%%%%%%%%%%%%% CUSTOM DOCUMENT %%%%%%%%%%%%%%%%%%%%
%%%%%%%%%%%%5%%%%%%%%%%%%%%%%%%%%%%%%%%%%%%%%%%%%%%%%%%%%%
% title pages
\usepackage{outline} 
\usepackage{pmgraph} 
\usepackage[normalem]{ulem}
\usepackage{verbatim}
\usepackage{pgffor}
\usepackage{xstring}
\usepackage{xparse} % for advanced argument parsing
% Signature Fields for Agreement
\usepackage{textcomp} % for \TextField
\usepackage{ifthen}   % for conditional checks
\newcounter{nameCounter}

% ------------------------------------------------------------------------------
% \signField[<Printed Name>][<Width Factor>]: Generates a signature field
% ------------------------------------------------------------------------------
% DESCRIPTION:
% This command creates a structured signature field with labeled sections for:
% - Signature
% - Printed Name (customizable label)
% - Date
% - Initials
%
% The width of the entire signature field can be adjusted using the second argument.
% The proportions remain the same while scaling.
%
% USAGE:
%   \signField[<Printed Name>][<Width Factor>]
%   - <Printed Name> (Optional): Defaults to "Printed Name"
%   - <Width Factor>: A decimal value specifying the fraction of \textwidth
%
% EXAMPLES:
%   - \shortSignField           
%   - \signField                
%   - \signField[Full Name]     → Uses "Full Name" for the name label by their usage.
% ------------------------------------------------------------------------------
\newcommand{\signField}[1][Printed Name]{
    \stepcounter{nameCounter}
    \begin{tabbing}
        \hspace{0.3\textwidth} \= \hspace{0.35\textwidth} \= \hspace{0.20\textwidth} \= \hspace{0.15\textwidth} \kill
        \makebox[0.3\textwidth]{} \> \TextField[name=name\arabic{nameCounter},width=0.35\textwidth,bordercolor=0 0 0]{} \> \TextField[name=date\arabic{nameCounter},width=0.20\textwidth,bordercolor=0 0 0]{} \> \makebox[0.15\textwidth]{} \\[-8pt]
        \underline{\hspace{0.3\textwidth}} \> \underline{\hspace{0.35\textwidth}} \> \underline{\hspace{0.20\textwidth}} \> \underline{\hspace{0.15\textwidth}} \\ 
        \textit{Signature} \> #1 \> Date \> \makebox[0.15\textwidth][r]{\textbf{\textit{Initials}}}
    \end{tabbing}
}
\newcommand{\shortSignField}[1][Printed Name]{%
    \stepcounter{nameCounter}

    % Define overall width as a multiple of \textwidth
    \newlength{\overallWidth}
    \setlength{\overallWidth}{0.8\textwidth}  % Slightly reduced total width

    % Calculate proportional column widths
    \newlength{\colOne}
    \newlength{\colTwo}
    \newlength{\colThree}

    \setlength{\colOne}{0.3\overallWidth}
    \setlength{\colTwo}{0.4\overallWidth}
    \setlength{\colThree}{0.3\overallWidth}

    \begin{tabbing}
        \hspace{\colOne} \= \hspace{\colTwo} \= \hspace{\colThree} \kill
        \makebox[\colOne]{} \> 
        \TextField[name=name\arabic{nameCounter},width=\colTwo,bordercolor=0 0 0]{} \> 
        \generatedOn \\[-8pt]
        \underline{\hspace{\colOne}} \> 
        \underline{\hspace{\colTwo}} \> 
        \underline{\hspace{\colThree}} \\ 
        \textit{Signature} \> #1 \> Date
    \end{tabbing}
}

%%%%%%%%%%%%%%%%%%%%%%%%%%%%%%%%%%%%%%%%%%%%%%%%%%%%%%%%%
% Title Page
%%%%%%%%%%%%%%%%%%%%%%%%%%%%%%%%%%%%%%%%%%%%%%%%%%%%%%%%%
\newcommand{\convertarraytostring}[2]{
  \gdef#1{}
  \foreach \x in {#2} {
    \xdef#1{#1 \x \\}
  }
}
\newcommand{\customtitle}[7]{
    \title{
        \includegraphics[width=1.75in]{images/emblem.png} \\ % fixed Logo settings
        \vspace*{0.5in}
        #1 \\ 
        \vspace*{0.5in}
        \textbf{#2}
    }
    \author{
        #3 \\ 
        \vspace*{1pt} \\
        #4 \\
        \vspace*{0.5in} \\
        \textbf{#5} \\ 
        \vspace*{1pt} \\
        #6 \\ 
        \vspace*{1pt} \\
        \textbf{#7} \\ 
    }
    \date{\today} % fixed date settings
    \maketitle
    \newpage
}

% \convertarraytostring
% {\titleauthors}                                                                       % Authors
% {Jaehoon Song (lead), Manya Jain, Devika Papal, Yashman P Singh}
% \customtitle
% {Final Team Project: Process Book}                                                    % Title
% {Music Trends and Insights: Visualization of Genres, Artists, and Listener Behaviors} % Subtitle
% {Team: Vizualytics}                                                                   % Team name (e.g., "Team: Vizualytics")
% {\titleauthors}
% {CS-4460-B: Introduction to Information Visualization}                                % Course name (e.g., "CS-4460-B: Introduction to ...")
% {Georgia Institute of Technology}                                                     % Institution (e.g., "Georgia Institute of Technology")
% {Professor: Yalong Yang}                                                              % Professor name (e.g., "Yalong Yang")












%%%%%%%%%%%%%%%%%%%%%%%%%%%%%%%%%%%%%%%%%%%%%%%%%%%%%%%%%
% Handout Title (First Page) 
%%%%%%%%%%%%%%%%%%%%%%%%%%%%%%%%%%%%%%%%%%%%%%%%%%%%%%%%%

% General Handout Format
% \handout{<title>}{<course name>}{<date>}{<author>}{<secondary info>}{<number>}
% Creates a formatted header for handouts. Parameters:
% 1. <is_shadow>: is the box shadowed
% 1. <title>: Title of the handout (e.g., "Lecture 1")
% 2. <course name>: Name of the course (e.g., "CS 4510 Automata and Complexity")
% 3. <author>: Name of the author (e.g., "Prof. Alice")
% 4. <date>: Date of the handout (e.g., "Jan 15, 2025")
% 5. <secondary info>: Additional info (e.g., "Scribe(s): Bob and Charlie")
% 6. <number>: Handout number or identifier (used for page numbering)
\usepackage{amsmath,amsfonts,amssymb,tikz} % Math symbols and environments
\usepackage{fancybox}
\usetikzlibrary{automata, positioning, arrows}
\usetikzlibrary{arrows.meta}
\newcommand{\handout}[7][false]{%
    \renewcommand{\thepage}{#7-\arabic{page}} % Sets page numbering style to <number>-<page>
    \noindent
    \begin{center}
    \ifthenelse{\equal{#1}{true}}{%
        % Use shadowbox if the boolean is true
        \shadowbox{%
            \parbox{5.78in}{%
                \vbox{
                    \hbox to 5.78in { #3 \hfill #4 } % Course name and date
                    \vspace{4mm}
                    \hbox to 5.78in { {\Large \hfill #2 \hfill} } % Title centered
                    \vspace{2mm}
                    \hbox to 5.78in { {\it #5 \hfill #6} } % Author and secondary info
                }
            }
        }
    }{%
        % Use standard framebox if the boolean is false
        \framebox{%
            \vbox{
                \hbox to 5.78in { #3 \hfill #4 } % Course name and date
                \vspace{4mm}
                \hbox to 5.78in { {\Large \hfill #2 \hfill} } % Title centered
                \vspace{2mm}
                \hbox to 5.78in { {\it #5 \hfill #6} } % Author and secondary info
            }
        }
    }
    \end{center}
    \vspace*{4mm}
}


% Example Usages:

% \handout
%    {Lecture 1}
%    {CS 4510 Automata and Complexity}{Prof. Alice}
%    {Jan 15, 2025}{Scribe(s): Bob and Charlie}
%    {1}

% \handout
% {Homework 1: Finite Automata}
% {\textbf{CS 4510 Automata and Complexity}}{Out: 01/08/2025}
% { }{Due: 01/22/2025}
% {1: Finite Automata}

% \handout[true]
% {Problem Set 1} 
% {\textsc{Georgia Tech} S'25}{CS 4540: \textsc{Advanced Algorithms}}
% {Out: Jan 7}{Due: Jan 14 11.59 pm}
% {CS 4540}


% \handout
% {Exam 1: Regular Languages}
% {\textbf{CS 4510 Automata and Complexity}}{Due: 12:30 pm February 6}
% {Released: 12:30 pm February 5}{ }
% {1: Regular Languages}


% \handout
%    {Project Proposal}
%    {CS 4510 Automata and Complexity}{Prof. Alice}
%    {Mar 1, 2025}{Team: Alpha Group}
%    {1}

% \handout
%    {Midterm Announcement}
%    {CS 4510 Automata and Complexity}{Prof. Alice}
%    {Mar 10, 2025}{--}
%    {--}











































%%%%%%%%%%%%%%%%%%%%%%%%%%%%%%%%%%%%%%%%%%%%%%%%%%%%%%%%%
% Mathematics
%%%%%%%%%%%%%%%%%%%%%%%%%%%%%%%%%%%%%%%%%%%%%%%%%%%%%%%%%

\newtheorem{definition}{Definition}











% ------------------------------------------------------------------------------
% \begin{subanswer}[<Color>][<isWhite>] … \end{subanswer}
% ------------------------------------------------------------------------------
% DESCRIPTION:
%   This custom environment displays an “Answer:” label with configurable text color.
%   It is intended for formatting answers clearly in assignments or solutions.
%
% FEATURES:
%   - Displays “Answer:” in bold, followed by answer content.
%   - Automatically ends with a square symbol (□) aligned right.
%   - First optional argument selects text color (default = red).
%   - Second optional argument overrides to white text when set to true.
%
% PARAMETERS (both optional):
%   [<Color>]    Text color name (e.g. blue, green). Defaults to red.
%   [<isWhite>]  true  ⇒ force white text (for dark backgrounds)
%                 false ⇒ use the chosen <Color>. Defaults to false.
%
% DEPENDENCIES:
%   \usepackage{ifthen}
%   \usepackage{xcolor}
%   \usepackage{xparse}
%
% USAGE EXAMPLES:
%   % defaults: red text, isWhite = false
%   \begin{subanswer}
%     This answer is red.
%   \end{subanswer}
%
%   % blue text, isWhite = false
%   \begin{subanswer}[blue]
%     This answer is blue.
%   \end{subanswer}
%
%   % green text, forced white override
%   \begin{subanswer}[green][true]
%     This answer is white (on green background).
%   \end{subanswer}
% ------------------------------------------------------------------------------
\NewDocumentEnvironment{subanswer}{ O{red} O{false} }
  {%
    \par\medskip                     % start a paragraph with a bit of space
    % choose white if second optional arg is “true”, otherwise #1
    \ifthenelse{\equal{#2}{true}}%
      {\color{white}}%
      {\color{#1}}%
    \noindent\textbf{Answer}:        % no \\ or \newline here
  }
  {%
    \par\hfill$\square$\par
  }
\NewDocumentEnvironment{subprove}{ O{red} O{false} }
  {%
    \par\medskip                     % start a paragraph with a bit of space
    % choose white if second optional arg is “true”, otherwise #1
    \ifthenelse{\equal{#2}{true}}%
      {\color{white}}%
      {\color{#1}}%
    \noindent\textbf{Proof}:        % no \\ or \newline here
  }
  {%
    \par\hfill$\blacksquare$\par
  }


% Answer environment for responses
% \answer[<color>] creates an answer environment with optional text color (default: teal).
\newenvironment{answer}[1][redcolor]
    {\ \newline \color{#1}\noindent\textbf{Answer}: }
    {\par\hfill$\square$\newline\newpage}
    % {\par\hfill$\square$\newline}

% Prove environment for responses
% \prove[<color>] creates an answer environment with optional text color (default: teal).
\newenvironment{prove}[1][redcolor]
    {\ \newline \color{#1}\noindent\textbf{Proof}: }
    {\par\hfill$\blacksquare$\newline\newpage}
    % {\par\hfill$\blacksquare$\newline}

% Define environment for framed definitions
% \define[<color>] creates a definition/theorem environment with a colored frame and background
\usepackage{mdframed}
\newenvironment{define}[2][bluecolor]
    {
    \vspace{2pt}
    \begin{mdframed}[backgroundcolor=#1!10, linecolor=#1]
    \color{#1}
    \textbf{\S~#2}\hspace{6pt}  % Start symbol for definition/theorem with a title
    \vspace{4pt}
    }
{
    \end{mdframed}
}



% Example Usage:

% To use the \answer environment:
% \begin{answer}
%   <YOUR WORK GOES HERE>
% \end{answer}
% \begin{answer}[red]
% This is a custom answer in green color.
% \end{answer}

% To use the \prove environment:
% \begin{prove}
%   <YOUR WORK GOES HERE>
% \end{prove}
% \begin{prove}[purple]
% This is a custom proof in purple color.
% \end{prove}

% To use the \define environment:
% \begin{define}{Name of Definition/Theorem/Property}
%   <YOUR Theorem/Definition GOES HERE>
% \end{define}
% \begin{define}[orange]{Name of Definition/Theorem/Property}
% This is a custom definition with an orange frame and background.
% \end{define}

% Passing a custom color example:
% Replace <color> with a valid LaTeX color name (e.g., red, green, blue, cyan, etc.)
% or define your own colors using \usepackage{xcolor} if needed.







% % Commonly used math symbols
% \newcommand{\T}{\mathrm{T}}
% \newcommand{\F}{\mathrm{F}}
% \newcommand{\Z}{\mathbb{Z}}
% \newcommand{\N}{\mathbb{N}}
% \newcommand{\R}{\mathbb{R}}
% \renewcommand{\O}{\mathcal{O}}




%%%%%%%%%%%%%%%%%%%%%%%%%%%%%%%%%%%%%%%%%%%%%%%%%%%%%%%%%
% Bibliographies (with Citations)
%%%%%%%%%%%%%%%%%%%%%%%%%%%%%%%%%%%%%%%%%%%%%%%%%%%%%%%%%
\usepackage[backend=biber,style=apa]{biblatex}      % \newpage\printbibliography\end{document}
\addbibresource{../references.bib}                      % \cite{reference1}
% \addbibresource{references.bib}                         % \cite{reference1}
%%%%%%%%%%%%%%%%%%%%%%%%%%%%%%%%%%%%%%%%%%%%%%%%%%%%%%%%%
% DEFAULT
%%%%%%%%%%%%%%%%%%%%%%%%%%%%%%%%%%%%%%%%%%%%%%%%%%%%%%%%%
                                                    % \textbf{boldfaced}
                                                    % \textit{intalicized}
%%%% MATH underbrace
% \[
%     a + b + c = \underbrace{x + y + z
%         }_{\text{
%         Sum of terms
%         }
%         }
% \]
\usepackage{enumitem}


























%%macros for Alg and Uncertainty
\newcommand\p{\mbox{\bf P}\xspace}
\newcommand\np{\mbox{\bf NP}\xspace}
\newcommand{\Alg}{\text{Alg}}
\newcommand{\Opt}{\text{Opt}}

\newcommand{\one}{\mathbf{1}\xspace}
\newcommand{\calD}{\mathcal{D}}
\newcommand{\F}{\mathcal{F}}
\newcommand{\calF}{\mathcal{F}}
\newcommand{\M}{\mathcal{M}}
\newcommand{\reals}{\mathbb{R}}
\newcommand{\sse}{\subseteq}
\newcommand{\Clients}{\text{Clients}}
\newcommand{\X}{\mathbf{X}}
\newcommand{\Y}{\mathbf{Y}}
\newcommand{\I}{\mathbb{I}}
\newcommand{\cc}{\mathbf{c}}
\def \integers {\mathbb{Z}}

\newcommand\OPT{\text{OPT}\xspace}


%%Computational problems

\newcommand\sat{\mbox{SAT}\xspace}

\newcommand\numsat{\mbox{$\sharp$ SAT}\xspace}
%% Notation for integers, natural numbers, reals, fractions, sets, cardinalities
%%and so on

\newcommand{\IGNORE}[1]{}
\newcommand\bz{\mbox{\bf Z}}
\newcommand{\R}{\mathbb{R}}
\newcommand{\E}{\mathbb{E}}
\renewcommand{\P}{\mathbb{P}}
\newcommand\nat{\mbox{\bf N}}
\newcommand\rea{\mbox{\bf R}}
\newcommand\B{\{0,1\}}      % boolean alphabet  use in math mode
\newcommand\Bs{\{0,1\}^*}   % B star            use in math mode
\newcommand\true{\mbox{\sc True}}
\newcommand\false{\mbox{\sc False}}
\DeclareRobustCommand{\fracp}[2]{{#1 \overwithdelims()#2}}
\DeclareRobustCommand{\fracb}[2]{{#1 \overwithdelims[]#2}}
\newcommand{\marginlabel}[1]%
{\mbox{}\marginpar{\it{\raggedleft\hspace{0pt}#1}}}
\newcommand\card[1]{\left| #1 \right|} %cardinality of set S; usage \card{S}
\newcommand\set[1]{\left\{#1\right\}} %usage \set{1,2,3,,}
\newcommand\poly{\mbox{poly}}  %usage \poly(n)
\DeclareMathOperator*{\argmin}{arg\,min}
\DeclareMathOperator*{\argmax}{arg\,max}
\DeclareMathOperator*{\sign}{sign}
\DeclareMathOperator*{\rank}{rank}
\newcommand{\floor}[1]{\lfloor\, $#1$\,\rfloor}
\newcommand{\ceil}[1]{\lceil\, $#1$\,\rceil}
\newcommand{\comp}[1]{\overline{#1}}
\newcommand{\defeq}{\overset{\mathrm{def}}{=}}
\newcommand{\bits}{\set{0,1}}



% Ordinals
% \newcommand{\st}{\textsuperscript{st}\ }
% \newcommand{\nd}{\textsuperscript{nd}\ }
% \newcommand{\rd}{\textsuperscript{rd}\ }
\newcommand{\thsup}{\textsuperscript{th}\ }


\usepackage{qtree} % `\Tree` for simple tree graphing!


\newenvironment{algo}{\vspace{0.07in} \noindent \begin{minipage}	{5.5in}%\begin{quote}
\hrule\vspace{0.01in}\hrule \vspace{0.05in}}{\vspace{-0.0in}\hrule\vspace{0.01in}\hrule \vspace{0.07in}%\end{quote} 
\end{minipage}}%


\usepackage{arydshln} % For dashed/dotted lines
\usepackage{cellspace} % Extra padding in table cells
\usepackage{multicol}

\usepackage{xspace}
\usepackage{braket}





\usepackage{algorithm}
\usepackage{algorithmic}
% \newcommand{\algorithmiccomment}[1]{\quad$\triangleright$ #1}
\makeatletter
\renewcommand{\algorithmiccomment}[1]{\hfill$\triangleright$~#1}
\makeatother

% \makeatletter ... \makeatother
% This pair temporarily makes @ a "letter" 
% so you can redefine internals (macros whose names contain @ ). 
% \algorithmiccomment internally uses @, so you need this wrapper.







% % ─── Enter “@‑letter” mode ────────────────────────────────
% \makeatletter

%   % Define internal macro \@sayhello that takes one argument
%   \def\@sayhello#1{Hello, #1!}

%   % Define user‑level \sayhello that simply calls the internal one
%   \newcommand{\sayhello}[1]{\@sayhello{#1}}

% % ─── Leave “@‑letter” mode ───────────────────────────────
% \makeatother

% % ─── Now use your friendly command in the document ───────
% \sayhello{World}





% ------------------------------------------------------------------------------
% \begin{state}[state#]{title text} 
%     ... your content ... 
% \end{state}
% : formats a subsection title.
% ------------------------------------------------------------------------------
% DESCRIPTION: 
%   Define the custom environment "state"
%   This environment formats a subsection title in the following style:
%   ```
%   §<number> (<title>)
%   ```
%
% ARGUMENTS:
%   - An optional argument: a number to which the counter should be reset.
%       If provided, the counter "statecounter" is set to (#1 - 1) so that after
%       incrementing, the displayed number is exactly the value provided.
%   - A mandatory argument: the title text.
%
% USAGE:
%   \begin{state}{title text}
%       ... your content ...
%   \end{state}
%
%   \begin{state}[<number>]{title text}
%       ... your content ...
%   \end{state}
%
% Example:
%   \begin{state}{title1}  % Displays: §1 (title1)
%   \end{state}
%
%   \begin{state}[2]{title4}  % Resets counter; displays: §2 (title4)
%   \end{state}
% ------------------------------------------------------------------------------
\newcounter{statecounter} % The counter "statecounter" is used to number the subsections.
\NewDocumentEnvironment{state}{ o m }{%
  % Check if the optional argument is provided.
  % If yes, reset the counter to (value - 1) so that after incrementing it equals the provided value.
  \IfNoValueTF{#1}{%
    % No optional argument provided; increment normally.
    \stepcounter{statecounter}%
  }{%
    % Optional argument provided; reset counter accordingly.
    \setcounter{statecounter}{\numexpr#1-1\relax}%
    \stepcounter{statecounter}%
  }%
  % Create an unnumbered subsection with our custom format:
  % "§<current counter> (<title text>)"
  \subsection*{§\arabic{statecounter} (#2)}%
}{%
  % End of the "state" environment.
}

% ------------------------------------------------------------------------------
% Define the alias environment "prop" as a wrapper for "state".
%
% This allows you to use either \begin{state} ... \end{state} or
% \begin{prop} ... \end{prop} with identical behavior.
% ------------------------------------------------------------------------------
\NewDocumentEnvironment{prop}{ o m }{%
  % Pass the arguments to the "state" environment.
  \IfNoValueTF{#1}{%
    \begin{state}{#2}%
  }{%
    \begin{state}[#1]{#2}%
  }%
}{%
  \end{state}%
}