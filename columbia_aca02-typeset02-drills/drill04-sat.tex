\newpage\handout
{Drill Problems: Week 2.6}
{\textsc{Scholastic Aptitude Test (SAT)}}
{\href{https://creativecommons.org/licenses/by-nc-sa/4.0/}{CC BY-NC-SA 4.0 license}}
{Author: \BookAuthor}{Release: \generatedOn}
{SAT: Drill Problems (2.6)}
\input{drill00-disclaimer.tex}

\begin{enumerate}
  \item \textbf{Rectangle Area Function} (10 points)\\
  A rectangle has a length that is 15 times its width. The function $y=(15w)(w)$ represents this situation, where $y$ is the area, in square feet, of the rectangle and $y>0$. Which of the following is the best interpretation of $15w$ in this context?
  \begin{enumerate}[label=(\Alph*)]
    \item The length of the rectangle, in feet
    \item The area of the rectangle, in square feet
    \item The difference between the length and the width of the rectangle, in feet
    \item The width of the rectangle, in feet
  \end{enumerate}
  \begin{subanswer}
    % your answer here
  \end{subanswer}

  \item \textbf{Quadratic Function Roots} (10 points)\\
  The quadratic function $h$ is defined as shown.
  \[
  h(x)=2(x-4)^{2}-32
  \]
  In the $xy$-plane, the graph of $y=h(x)$ intersects the $x$-axis at the points $(0,0)$ and $(t,0)$, where $t$ is a constant.
  
  What is the value of $t$?
  \begin{enumerate}[label=(\Alph*)]
    \item 1
    \item 2
    \item 4
    \item 8
  \end{enumerate}
  \begin{subanswer}
    % your answer here
  \end{subanswer}

  \item \textbf{Exponential Function Y-Intercept} (10 points)\\
  The function $f$ is defined by $f(x)=(-8)(2)^{x}+22$. What is the $y$-intercept of the graph of $y=f(x)$ in the $xy$-plane?
  \begin{enumerate}[label=(\Alph*)]
    \item $(0,14)$
    \item $(0,2)$
    \item $(0,22)$
    \item $(0,-8)$
  \end{enumerate}
  \begin{subanswer}
    % your answer here
  \end{subanswer}

  \newpage

  \item \textbf{Ball Height Interpretation} (10 points)\\
  \insertimage{0.40}{images/2025_06_15_5ccb8dd1752fe76599e7g-004}{reference attached}
  The graph shows the height above ground, in meters, of a ball $x$ seconds after the ball was launched upward from a platform. Which statement is the best interpretation of the marked point $(1.0, 4.8)$ in this context?
  \begin{enumerate}[label=(\Alph*)]
    \item 1.0 second after being launched, the ball's height above ground is 4.8 meters.
    \item 4.8 seconds after being launched, the ball's height above ground is 1.0 meter.
    \item The ball was launched from an initial height of 1.0 meter with an initial velocity of 4.8 meters per second.
    \item The ball was launched from an initial height of 4.8 meters with an initial velocity of 1.0 meter per second.
  \end{enumerate}
  \begin{subanswer}
    % your answer here
  \end{subanswer}

  \newpage

  \item \textbf{Exponential Decay Y-Intercept} (10 points)\\
  The given function $f$ models the number of advertisements a company sent to its clients each year, where $x$ represents the number of years since 1997, and $0 \leq x \leq 5$.
  \[
  f(x)=9,000(0.66)^{x}
  \]
  If $y=f(x)$ is graphed in the $xy$-plane, which of the following is the best interpretation of the $y$-intercept of the graph in this context?
  \begin{enumerate}[label=(\Alph*)]
    \item The minimum estimated number of advertisements the company sent to its clients during the 5 years was 1,708.
    \item The minimum estimated number of advertisements the company sent to its clients during the 5 years was 9,000.
    \item The estimated number of advertisements the company sent to its clients in 1997 was 1,708.
    \item The estimated number of advertisements the company sent to its clients in 1997 was 9,000.
  \end{enumerate}
  \begin{subanswer}
    % your answer here
  \end{subanswer}

  \item \textbf{Geometric Sequence Formula} (10 points)\\
  The first term of a sequence is 9. Each term after the first is 4 times the preceding term. If $w$ represents the $n$th term of the sequence, which equation gives $w$ in terms of $n$?
  \begin{enumerate}[label=(\Alph*)]
    \item $w=4\left(9^{n}\right)$
    \item $w=4\left(9^{n-1}\right)$
    \item $w=9\left(4^{n}\right)$
    \item $w=9\left(4^{n-1}\right)$
  \end{enumerate}
  \begin{subanswer}
    % your answer here
  \end{subanswer}

  \newpage

  \item \textbf{Graph Y-Intercept} (10 points)\\
  \insertimage{0.40}{images/2025_06_15_5ccb8dd1752fe76599e7g-007}{reference attached}
  What is the $y$-intercept of the graph shown?
  \begin{enumerate}[label=(\Alph*)]
    \item $(-1,-9)$
    \item $(0,-5)$
    \item $(0,-4)$
    \item $(0,0)$
  \end{enumerate}
  \begin{subanswer}
    % your answer here
  \end{subanswer}

  \newpage

  \item \textbf{Savings Account Exponential Model} (10 points)\\
  Rosa opened a savings account at a bank. The table shows the exponential relationship between the time $t$, in years, since Rosa opened the account and the total amount $n$, in dollars, in the account.
  \begin{table}[h!]
  \centering
  \renewcommand{\arraystretch}{1.3}
  \setlength{\tabcolsep}{8pt}
  \caption*{\textbf{Savings Account Balance}}
  \begin{tabular}{|c|c|}
  \hline
  \rowcolor[HTML]{E0E0E0}
  \textbf{Time (years)} & \textbf{Total amount (dollars)} \\
  \hline
  0 & 604.00 \\
  \hline
  1 & 606.42 \\
  \hline
  2 & 608.84 \\
  \hline
  \end{tabular}
  \end{table}
  If Rosa made no additional deposits or withdrawals, which of the following equations best represents the relationship between $t$ and $n$?
  \begin{enumerate}[label=(\Alph*)]
    \item $n=604(1.004)^t$
    \item $n=604(1.04)^t$
    \item $n=604(1.004)^{t+1}$
    \item $n=0.004(604)^t$
  \end{enumerate}
  \begin{subanswer}
    % your answer here
  \end{subanswer}

  \item \textbf{Exponential Function Parameters} (10 points)\\
  Function $f$ is defined by $f(x)=-a^{x}+b$, where $a$ and $b$ are constants. In the $xy$-plane, the graph of $y=f(x)-12$ has a $y$-intercept at $\left(0,-\frac{75}{7}\right)$. The product of $a$ and $b$ is $\frac{320}{7}$. What is the value of $a$?
  \begin{subanswer}
    % your answer here
  \end{subanswer}

  \newpage

  \item \textbf{Ocean Water Level Model} (10 points)\\
  \insertimage{0.40}{images/2025_06_15_5ccb8dd1752fe76599e7g-010}{reference attached}
  Scientists recorded data about the ocean water levels at a certain location over a period of 6 hours. The graph shown models the data, where $y=0$ represents sea level. Which table gives values of $x$ and their corresponding values of $y$ based on the model?
  \begin{center}
  \begin{tabular}{cccc}
  \begin{minipage}{0.22\textwidth}
  \begin{enumerate}[label=(\Alph*)]
  \item
  \renewcommand{\arraystretch}{1.3}
  \setlength{\tabcolsep}{8pt}
  \begin{tabular}{|c|c|}
  \hline
  $x$ & $y$ \\
  \hline
  0 & -12 \\
  \hline
  0 & 3 \\
  \hline
  6 & 6 \\
  \hline
  \end{tabular}
  \end{enumerate}
  \end{minipage}
  &
  \begin{minipage}{0.22\textwidth}
  \begin{enumerate}[label=(\Alph*)]
  \item
  \renewcommand{\arraystretch}{1.3}
  \setlength{\tabcolsep}{8pt}
  \begin{tabular}{|c|c|}
  \hline
  $x$ & $y$ \\
  \hline
  0 & 0 \\
  \hline
  3 & 12 \\
  \hline
  0 & -6 \\
  \hline
  \end{tabular}
  \end{enumerate}
  \end{minipage}
  &
  \begin{minipage}{0.22\textwidth}
  \begin{enumerate}[label=(\Alph*)]
  \item
  \renewcommand{\arraystretch}{1.3}
  \setlength{\tabcolsep}{8pt}
  \begin{tabular}{|c|c|}
  \hline
  $x$ & $y$ \\
  \hline
  0 & 0 \\
  \hline
  3 & -12 \\
  \hline
  6 & 0 \\
  \hline
  \end{tabular}
  \end{enumerate}
  \end{minipage}
  &
  \begin{minipage}{0.22\textwidth}
  \begin{enumerate}[label=(\Alph*)]
  \item
  \renewcommand{\arraystretch}{1.3}
  \setlength{\tabcolsep}{8pt}
  \begin{tabular}{|c|c|}
  \hline
  $x$ & $y$ \\
  \hline
  0 & 0 \\
  \hline
  12 & 6 \\
  \hline
  -6 & 0 \\
  \hline
  \end{tabular}
  \end{enumerate}
  \end{minipage}
  \end{tabular}
  \end{center}
  \begin{subanswer}
    % your answer here
  \end{subanswer}

  \item \textbf{Square Root Function Evaluation} (10 points)\\
  The function $f$ is defined by $f(x)=4+\sqrt{x}$. What is the value of $f(144)$?
  \begin{enumerate}[label=(\Alph*)]
    \item 0
    \item 16
    \item 40
    \item 76
  \end{enumerate}
  \begin{subanswer}
    % your answer here
  \end{subanswer}


  \newpage

  \item \textbf{Rectangular Court Dimensions} (10 points)\\
  A rectangular volleyball court has an area of 162 square meters. If the length of the court is twice the width, what is the width of the court, in meters?
  \begin{enumerate}[label=(\Alph*)]
    \item 9
    \item 18
    \item 27
    \item 54
  \end{enumerate}
  \begin{subanswer}
    % your answer here
  \end{subanswer}

  \item \textbf{Softball Height Equation} (10 points)\\
  A machine launches a softball from ground level. The softball reaches a maximum height of 51.84 meters above the ground at 1.8 seconds and hits the ground at 3.6 seconds. Which equation represents the height above ground $h$, in meters, of the softball $t$ seconds after it is launched?
  \begin{enumerate}[label=(\Alph*)]
    \item $h=-t^{2}+3.6$
    \item $h=-t^{2}+51.84$
    \item $h=-64(t-2.7)^{2}+51.84$
    \item $h=-16t^{2}+57.6t$
  \end{enumerate}
  \begin{subanswer}
    % your answer here
  \end{subanswer}

  \item \textbf{Exponential Function Intercepts} (10 points)\\
  The function $f$ is defined by $f(x)=a^{x}+b$, where $a$ and $b$ are constants. In the $xy$-plane, the graph of $y=f(x)$ has an $x$-intercept at $(2,0)$ and a $y$-intercept at $(0,-323)$. What is the value of $b$?
  \begin{subanswer}
    % your answer here
  \end{subanswer}

  \item \textbf{Salary Growth Model} (10 points)\\
  The function $S$ above models the annual salary, in dollars, of an employee $n$ years after starting a job, where $a$ is a constant.
  \[
  S(n)=38,000 a^{n}
  \]
  If the employee's salary increases by $4\%$ each year, what is the value of $a$?
  \begin{enumerate}[label=(\Alph*)]
    \item 0.04
    \item 0.4
    \item 1.04
    \item 1.4
  \end{enumerate}
  \begin{subanswer}
    % your answer here
  \end{subanswer}

  \item \textbf{Revenue Function Interpretation} (10 points)\\
  The revenue $f(x)$, in dollars, that a company receives from sales of a product is given by the function $f$ above, where $x$ is the unit price, in dollars, of the product.
  \[
  f(x)=-500 x^{2}+25\,000 x
  \]
  The graph of $y=f(x)$ in the $xy$-plane intersects the $x$-axis at 0 and $a$. What does $a$ represent?
  \begin{enumerate}[label=(\Alph*)]
    \item The revenue, in dollars, when the unit price of the product is \$0
    \item The unit price, in dollars, of the product that will result in maximum revenue
    \item The unit price, in dollars, of the product that will result in a revenue of \$0
    \item The maximum revenue, in dollars, that the company can make
  \end{enumerate}
  \begin{subanswer}
    % your answer here
  \end{subanswer}

  \item \textbf{Bacteria Growth Prediction} (10 points)\\
  A culture of bacteria is growing at an exponential rate, as shown in the table above.
  \begin{table}[h!]
  \centering
  \renewcommand{\arraystretch}{1.3}
  \setlength{\tabcolsep}{8pt}
  \caption*{\textbf{Growth of a Culture of Bacteria}}
  \begin{tabular}{|c|c|}
  \hline
  \rowcolor[HTML]{E0E0E0}
  \textbf{Day} & \begin{tabular}{c}
  \textbf{Number of bacteria per} \\
  \textbf{milliliter at end of day} \\
  \end{tabular} \\
  \hline
  1 & $2.5 \times 10^{5}$ \\
  \hline
  2 & $5.0 \times 10^{5}$ \\
  \hline
  3 & $1.0 \times 10^{6}$ \\
  \hline
  \end{tabular}
  \end{table}
  At this rate, on which day would the number of bacteria per milliliter reach $5.12 \times 10^{8}$?
  \begin{enumerate}[label=(\Alph*)]
    \item Day 5
    \item Day 9
    \item Day 11
    \item Day 12
  \end{enumerate}
  \begin{subanswer}
    % your answer here
  \end{subanswer}

  \newpage

  \item \textbf{Data Traffic Model Interpretation} (10 points)\\
  The equation above estimates the global data traffic $D$, in terabytes, for the year that is $t$ years after 2010.
  \[
  D=5,640(1.9)^{t}
  \]
  What is the best interpretation of the number 5,640 in this context?
  \begin{enumerate}[label=(\Alph*)]
    \item The estimated amount of increase of data traffic, in terabytes, each year
    \item The estimated percent increase in the data traffic, in terabytes, each year
    \item The estimated data traffic, in terabytes, for the year that is $t$ years after 2010
    \item The estimated data traffic, in terabytes, in 2010
  \end{enumerate}
  \begin{subanswer}
    % your answer here
  \end{subanswer}

  \item \textbf{Quadratic Function Properties} (10 points)\\
  In the given quadratic function, $a$ and $c$ are constants. The graph of $y=f(x)$ in the $xy$-plane is a parabola that opens upward and has a vertex at the point $(h, k)$, where $h$ and $k$ are constants.
  \[
  f(x)=a x^{2}+4 x+c
  \]
  If $k<0$ and $f(-9)=f(3)$, which of the following must be true?
  \begin{enumerate}[label=(\Roman*)]
    \item $c<0$
    \item $a \geq 1$
  \end{enumerate}
  \begin{enumerate}[label=(\Alph*)]
    \item I only
    \item II only
    \item I and II
    \item Neither I nor II
  \end{enumerate}
  \begin{subanswer}
    % your answer here
  \end{subanswer}

  \item \textbf{Exponent Rules} (10 points)\\
  Which expression is equivalent to $\left(m^{4} q^{4} z^{-1}\right)\left(m q^{5} z^{3}\right)$, where $m$, $q$, and $z$ are positive?
  \begin{enumerate}[label=(\Alph*)]
    \item $m^{4} q^{20} z^{-3}$
    \item $m^{5} q^{9} z^{2}$
    \item $m^{6} q^{8} z^{-1}$
    \item $m^{20} q^{12} z^{-2}$
  \end{enumerate}
  \begin{subanswer}
    % your answer here
  \end{subanswer}

  \item \textbf{Polynomial Factoring} (10 points)\\
  Which of the following is a factor of the polynomial above?
  \[
  4 a^{2}+20 a b+25 b^{2}
  \]
  \begin{enumerate}[label=(\Alph*)]
    \item $a+b$
    \item $2 a+5 b$
    \item $4 a+5 b$
    \item $4 a+25 b$
  \end{enumerate}
  \begin{subanswer}
    % your answer here
  \end{subanswer}

  \item \textbf{Polynomial Operations} (10 points)\\
  If $p=3 x+4$ and $v=x+5$, which of the following is equivalent to $p v-2 p+v$?
  \begin{enumerate}[label=(\Alph*)]
    \item $3 x^{2}+12 x+7$
    \item $3 x^{2}+14 x+17$
    \item $3 x^{2}+19 x+20$
    \item $3 x^{2}+26 x+33$
  \end{enumerate}
  \begin{subanswer}
    % your answer here
  \end{subanswer}

  \item \textbf{Linear Expression Simplification} (10 points)\\
  Which of the following is equivalent to the given expression?
  \[
  (x+5)+(2 x-3)
  \]
  \begin{enumerate}[label=(\Alph*)]
    \item $3 x-2$
    \item $3 x+2$
    \item $3 x-8$
    \item $3 x+8$
  \end{enumerate}
  \begin{subanswer}
    % your answer here
  \end{subanswer}

  \newpage

  \item \textbf{Rational Expression Simplification} (10 points)\\
  Which expression is equivalent to $\frac{8 x(x-7)-3(x-7)}{2 x-14}$, where $x>7$?
  \begin{enumerate}[label=(\Alph*)]
    \item $\frac{x-7}{5}$
    \item $\frac{8 x-3}{2}$
    \item $\frac{8 x^{2}-3 x-14}{2 x-14}$
    \item $\frac{8 x^{2}-3 x-77}{2 x-14}$
  \end{enumerate}
  \begin{subanswer}
    % your answer here
  \end{subanswer}

  \item \textbf{Polynomial Factoring} (10 points)\\
  Which of the following is equivalent to the expression $x^{4}-x^{2}-6$?
  \begin{enumerate}[label=(\Alph*)]
    \item $\left(x^{2}+1\right)\left(x^{2}-6\right)$
    \item $\left(x^{2}+2\right)\left(x^{2}-3\right)$
    \item $\left(x^{2}+3\right)\left(x^{2}-2\right)$
    \item $\left(x^{2}+6\right)\left(x^{2}-1\right)$
  \end{enumerate}
  \begin{subanswer}
    % your answer here
  \end{subanswer}

  \item \textbf{Polynomial Expansion} (10 points)\\
  Which of the following is equivalent to the expression above?
  \[
  (2 x+5)^{2}-(x-2)+2(x+3)
  \]
  \begin{enumerate}[label=(\Alph*)]
    \item $4 x^{2}+21 x+33$
    \item $4 x^{2}+21 x+29$
    \item $4 x^{2}+x+29$
    \item $4 x^{2}+x+33$
  \end{enumerate}
  \begin{subanswer}
    % your answer here
  \end{subanswer}

  \newpage

  \item \textbf{Polynomial Multiplication} (10 points)\\
  The equation above is true for all $x$, where $a$ and $b$ are constants.
  \[
  (a x+3)\left(5 x^{2}-b x+4\right)=20 x^{3}-9 x^{2}-2 x+12
  \]
  What is the value of $a b$?
  \begin{enumerate}[label=(\Alph*)]
    \item 18
    \item 20
    \item 24
    \item 40
  \end{enumerate}
  \begin{subanswer}
    % your answer here
  \end{subanswer}

  \item \textbf{Difference of Squares} (10 points)\\
  Which of the following expressions is equivalent to $x^{2}-5$?
  \begin{enumerate}[label=(\Alph*)]
    \item $(x+\sqrt{5})^{2}$
    \item $(x-\sqrt{5})^{2}$
    \item $(x+\sqrt{5})(x-\sqrt{5})$
    \item $(x+5)(x-1)$
  \end{enumerate}
  \begin{subanswer}
    % your answer here
  \end{subanswer}

  \item \textbf{Quadratic Factoring} (10 points)\\
  Which of the following expressions is(are) a factor of $3 x^{2}+20 x-63$?
  \begin{enumerate}[label=(\Roman*)]
    \item $x-9$
    \item $3 x-7$
  \end{enumerate}
  \begin{enumerate}[label=(\Alph*)]
    \item I only
    \item II only
    \item I and II
    \item Neither I nor II
  \end{enumerate}
  \begin{subanswer}
    % your answer here
  \end{subanswer}

  \item \textbf{Rational Exponent Simplification} (10 points)\\
  If $\frac{\sqrt{x^{5}}}{\sqrt[3]{x^{4}}}=x^{\frac{a}{b}}$ for all positive values of $x$, what is the value of $\frac{a}{b}$?
  \begin{subanswer}
    % your answer here
  \end{subanswer}

  \item \textbf{Polynomial Factoring} (10 points)\\
  The expression $90 y^{5}-54 y^{4}$ is equivalent to $r y^{4}(15 y-9)$, where $r$ is a constant. What is the value of $r$?
  \begin{subanswer}
    % your answer here
  \end{subanswer}

  \item \textbf{Rational Equation} (10 points)\\
  The equation above is true for all $x>2$, where $r$ and $t$ are positive constants.
  \[
  \frac{2}{x-2}+\frac{3}{x+5}=\frac{x+t}{(x-2)(x+5)}
  \]
  What is the value of $r t$?
  \begin{enumerate}[label=(\Alph*)]
    \item -20
    \item 15
    \item 20
    \item 60
  \end{enumerate}
  \begin{subanswer}
    % your answer here
  \end{subanswer}

  \item \textbf{Polynomial Simplification} (10 points)\\
  Which of the following is an equivalent form of $(1.5 x-2.4)^{2}-\left(5.2 x^{2}-6.4\right)$?
  \begin{enumerate}[label=(\Alph*)]
    \item $-2.2 x^{2}+1.6$
    \item $-2.2 x^{2}+11.2$
    \item $-2.95 x^{2}-7.2 x+12.16$
    \item $-2.95 x^{2}-7.2 x+0.64$
  \end{enumerate}
  \begin{subanswer}
    % your answer here
  \end{subanswer}

  \item \textbf{Root Expression Simplification} (10 points)\\
  For what value of $x$ is the given expression equivalent to $(70 n)^{30 x}$, where $n>1$?
  \[
  \sqrt[5]{70 n}(\sqrt[6]{70 n})^{2}
  \]
  \begin{subanswer}
    % your answer here
  \end{subanswer}

  \newpage

  \item \textbf{System of Equations Solutions} (10 points)\\
  \insertimage{0.30}{images/2025_06_15_6ec744ffc48635e98821g-01}{reference attached}
  A system of equations consists of a quadratic equation and a linear equation. The equations in this system are graphed in the $xy$-plane above. How many solutions does this system have?
  \begin{enumerate}[label=(\Alph*)]
    \item 0
    \item 1
    \item 2
    \item 3
  \end{enumerate}
  \begin{subanswer}
    % your answer here
  \end{subanswer}

  \item \textbf{Linear Inequality} (10 points)\\
  Which of the following inequalities is equivalent to the inequality above?
  \[
  6 x-9 y>12
  \]
  \begin{enumerate}[label=(\Alph*)]
    \item $x-y>2$
    \item $2 x-3 y>4$
    \item $3 x-2 y>4$
    \item $3 y-2 x>2$
  \end{enumerate}
  \begin{subanswer}
    % your answer here
  \end{subanswer}

  \newpage

  \item \textbf{System of Equations Solution} (10 points)\\
  If $(x, y)$ is a solution to the system of equations above, which of the following could be the value of $x$?
  \[
  \begin{aligned}
  & y=x+1 \\
  & y=x^{2}+x
  \end{aligned}
  \]
  \begin{enumerate}[label=(\Alph*)]
    \item -1
    \item 0
    \item 2
    \item 3
  \end{enumerate}
  \begin{subanswer}
    % your answer here
  \end{subanswer}

  \item \textbf{Quadratic Equation Solutions} (10 points)\\
  What values satisfy the equation above?
  \[
  x^{2}-x-1=0
  \]
  \begin{enumerate}[label=(\Alph*)]
    \item $x=1$ and $x=2$
    \item $x=-\frac{1}{2}$ and $x=\frac{3}{2}$
    \item $x=\frac{1+\sqrt{5}}{2}$ and $x=\frac{1-\sqrt{5}}{2}$
    \item $x=\frac{-1+\sqrt{5}}{2}$ and $x=\frac{-1-\sqrt{5}}{2}$
  \end{enumerate}
  \begin{subanswer}
    % your answer here
  \end{subanswer}

  \item \textbf{Function Intersection} (10 points)\\
  The graphs of the given equations intersect at the point $(x, y)$ in the $xy$-plane.
  \[
  \begin{gathered}
  x=49 \\
  y=\sqrt{x}+9
  \end{gathered}
  \]
  What is the value of $y$?
  \begin{enumerate}[label=(\Alph*)]
    \item 16
    \item 40
    \item 81
    \item 130
  \end{enumerate}
  \begin{subanswer}
    % your answer here
  \end{subanswer}

  \item \textbf{Absolute Value Equation} (10 points)\\
  What is the positive solution to the given equation?
  \[
  2|4-x|+3|4-x|=25
  \]
  \begin{subanswer}
    % your answer here
  \end{subanswer}

  \item \textbf{Quadratic Equation Solution} (10 points)\\
  One solution to the given equation can be written as $1+\sqrt{k}$, where $k$ is a constant.
  \[
  x^{2}-2 x-9=0
  \]
  What is the value of $k$?
  \begin{enumerate}[label=(\Alph*)]
    \item 8
    \item 10
    \item 20
    \item 40
  \end{enumerate}
  \begin{subanswer}
    % your answer here
  \end{subanswer}

  \item \textbf{Parabola and Line Intersection} (10 points)\\
  In the $xy$-plane, a line with equation $2 y=4.5$ intersects a parabola at exactly one point. If the parabola has equation $y=-4 x^{2}+b x$, where $b$ is a positive constant, what is the value of $b$?
  \begin{subanswer}
    % your answer here
  \end{subanswer}

  \item \textbf{System of Equations Solution} (10 points)\\
  Which ordered pair is a solution to the system of equations above?
  \[
  \begin{aligned}
  & x-y=1 \\
  & x+y=x^{2}-3
  \end{aligned}
  \]
  \begin{enumerate}[label=(\Alph*)]
    \item $(1+\sqrt{3}, \sqrt{3})$
    \item $(\sqrt{3},-\sqrt{3})$
    \item $(1+\sqrt{5}, \sqrt{5})$
    \item $(\sqrt{5},-1+\sqrt{5})$
  \end{enumerate}
  \begin{subanswer}
    % your answer here
  \end{subanswer}
  
  \newpage

  \item \textbf{Variable Isolation} (10 points)\\
  The given equation relates the variables $r$, $s$, and $t$. Which equation correctly expresses $s$ in terms of $r$ and $t$?
  \[
  6 r=7 s+t
  \]
  \begin{enumerate}[label=(\Alph*)]
    \item $s=42 r-t$
    \item $s=7(6 r-t)$
    \item $s=\frac{6}{7} r-t$
    \item $s=\frac{6 r-t}{7}$
  \end{enumerate}
  \begin{subanswer}
    % your answer here
  \end{subanswer}

  \item \textbf{Rational Equation Solution} (10 points)\\
  If $x$ is a solution to the given equation, which of the following is a possible value of $x+5$?
  \[
  \frac{1}{x^{2}+10 x+25}=4
  \]
  \begin{enumerate}[label=(\Alph*)]
    \item $\frac{1}{2}$
    \item $\frac{5}{2}$
    \item $\frac{9}{2}$
    \item $\frac{11}{2}$
  \end{enumerate}
  \begin{subanswer}
    % your answer here
  \end{subanswer}

  \item \textbf{Acceleration Equation} (10 points)\\
  During a 5-second time interval, the average acceleration $a$, in 
  meters per second squared, of an object with an initial velocity 
  of 12 meters per second is defined by the equation $$a=\frac{v_{f}-12}{5}$$, 
  where $v_f$ is the final velocity of the object in meters per second. If 
  the equation is rewritten in the form $v_f=x a+y$, where $x$ and $y$ are 
  constants, what is the value of $x$?
  \begin{subanswer}
    % your answer here
  \end{subanswer}

  \newpage

  \item \textbf{Function Intersection} (10 points)\\
  The graphs of the given equations in the $xy$-plane intersect at the point $(x, y)$.
  \[
  \begin{gathered}
  y=76 \\
  y=x^{2}-5
  \end{gathered}
  \]
  What is a possible value of $x$?
  \begin{enumerate}[label=(\Alph*)]
    \item $-\frac{76}{5}$
    \item -9
    \item 5
    \item 76
  \end{enumerate}
  \begin{subanswer}
    % your answer here
  \end{subanswer}

  \item \textbf{Variable Isolation} (10 points)\\
  The given equation relates the positive numbers $m$, $n$, and $p$. Which equation correctly gives $n$ in terms of $m$ and $p$?
  \[
  7 m=5(n+p)
  \]
  \begin{enumerate}[label=(\Alph*)]
    \item $n=\frac{5 p}{7 m}$
    \item $n=\frac{7 m}{5}-p$
    \item $n=5(7 m)+p$
    \item $n=7 m-5-p$
  \end{enumerate}
  \begin{subanswer}
    % your answer here
  \end{subanswer}

  \item \textbf{Quadratic Equation Solution} (10 points)\\
  Which of the following is a solution to the equation above?
  \[
  2 x^{2}-2=2 x+3
  \]
  \begin{enumerate}[label=(\Alph*)]
    \item 2
    \item $1-\sqrt{11}$
    \item $\frac{1}{2}+\sqrt{11}$
    \item $\frac{1+\sqrt{11}}{2}$
  \end{enumerate}
  \begin{subanswer}
    % your answer here
  \end{subanswer}

  \item \textbf{Parabola and Line Intersection} (10 points)\\
  In the $xy$-plane, a line with equation $2 y=c$ for some constant $c$ intersects a parabola at exactly one point. If the parabola has equation $y=-2 x^{2}+9 x$, what is the value of $c$?
  \begin{subanswer}
    % your answer here
  \end{subanswer}

  \newpage

  
  \item \textbf{System of Equations Solution} (10 points)\\
  \insertimage{0.40}{images/2025_06_15_6ec744ffc48635e98821g-17}{reference attached}
  The graph of a system of a linear equation and a nonlinear equation is shown. What is the solution $(x, y)$ to this system?
  \begin{enumerate}[label=(\Alph*)]
    \item $(0,0)$
    \item $(0,2)$
    \item $(2,4)$
    \item $(4,0)$
  \end{enumerate}
  \begin{subanswer}
    % your answer here
  \end{subanswer}

  \item \textbf{Polynomial Roots Product} (10 points)\\
  What is the product of the solutions to the given equation?
  \[
  (x-4)(x+2)(x-1)=0
  \]
  \begin{enumerate}[label=(\Alph*)]
    \item 8
    \item 3
    \item -3
    \item -8
  \end{enumerate}
  \begin{subanswer}
    % your answer here
  \end{subanswer}

  \item \textbf{Rational Equation Solution} (10 points)\\
  What is the solution to the equation above?
  \[
  \frac{2(x+1)}{x+5}=1-\frac{1}{x+5}
  \]
  \begin{enumerate}[label=(\Alph*)]
    \item 0
    \item 2
    \item 3
    \item 5
  \end{enumerate}
  \begin{subanswer}
    % your answer here
  \end{subanswer}

  \item \textbf{Radical Equation} (10 points)\\
  What is the smallest solution to the given equation?
  \[
  \sqrt{(x-2)^{2}}=\sqrt{3 x+34}
  \]
  \begin{subanswer}
    % your answer here
  \end{subanswer}
\end{enumerate}


