\newpage\handout
{Drill Problems 2}
{\textsc{Scholastic Aptitude Test (SAT)}}
{\href{https://creativecommons.org/licenses/by-nc-sa/4.0/}{CC BY-NC-SA 4.0 license}}
{Author: \BookAuthor}{Release: \generatedOn}
{SAT: Drill Problems (2)}
\input{drill00-disclaimer.tex}

\begin{enumerate}
  \item \textbf{Arc Length Calculation} (10 points)\\
  \insertimage{0.30}{images/2025_06_15_44388a8c3a9d04fd81aag-01}{Circle with center O and arcs}
  The circle above has center $O$, the length of arc $\overline{ADC}$ is $5\pi$, and $x=100$. What is the length of arc $\overline{ABC}$?
  \begin{enumerate}[label=(\Alph*)]
    \item $9\pi$
    \item $13\pi$
    \item $18\pi$
    \item $\frac{13}{2}\pi$
  \end{enumerate}
  \begin{subanswer}
    % your answer here
  \end{subanswer}

  \item \textbf{Circle Radius from Equation} (10 points)\\
  The graph of $x^{2}+x+y^{2}+y=\frac{199}{2}$ in the $xy$-plane is a circle. What is the length of the circle's radius?
  \begin{subanswer}
    % your answer here
  \end{subanswer}

  \item \textbf{Circle Center Coordinates} (10 points)\\
  The equation above defines a circle in the $xy$-plane. What are the coordinates of the center of the circle?
  \begin{enumerate}[label=(\Alph*)]
    \item $(-20,-16)$
    \item $(-10,-8)$
    \item $(10,8)$
    \item $(20,16)$
  \end{enumerate}
  \begin{subanswer}
    % your answer here
  \end{subanswer}

  \newpage

  \item \textbf{Arc Length with Angle} (10 points)\\
  \insertimage{0.40}{images/2025_06_15_44388a8c3a9d04fd81aag-04}{reference attached}
  Point $O$ is the center of the circle above, and the measure of $\angle OAB$ is $30^{\circ}$. If the length of $\overline{OC}$ is 18, what is the length of arc $\overline{AB}$?
  \begin{enumerate}[label=(\Alph*)]
    \item $9\pi$
    \item $12\pi$
    \item $15\pi$
    \item $18\pi$
  \end{enumerate}
  \begin{subanswer}
    % your answer here
  \end{subanswer}

  \item \textbf{Circle Radius from Diameter} (10 points)\\
  A circle in the $xy$-plane has a diameter with endpoints $(2,4)$ and $(2,14)$. An equation of this circle is $(x-2)^{2}+(y-9)^{2}=r^{2}$, where $r$ is a positive constant. What is the value of $r$?
  \begin{subanswer}
    % your answer here
  \end{subanswer}

  \newpage

  \item \textbf{Angle Calculation in Triangle} (10 points)\\
  \insertimage{0.40}{images/2025_06_15_d0312806c0eedc278ae1g-01}{reference attached}
  In the figure, $AC=CD$. The measure of angle $EBC$ is $45^{\circ}$, and the measure of angle $ACD$ is $104^{\circ}$. What is the value of $x$?
  \begin{subanswer}
    % your answer here
  \end{subanswer}

  \item \textbf{Complex Angle Calculation} (10 points)\\
  \insertimage{0.40}{images/2025_06_15_d0312806c0eedc278ae1g-02}{reference attached}
  In the figure shown, points $Q$, $R$, $S$, and $T$ lie on line segment $PV$, and line segment $RU$ intersects line segment $SX$ at point $W$. The measure of $\angle SQX$ is $48^{\circ}$, the measure of $\angle SXQ$ is $86^{\circ}$, the measure of $\angle SWU$ is $85^{\circ}$, and the measure of $\angle VTU$ is $162^{\circ}$. What is the measure, in degrees, of $\angle TUR$?
  \begin{subanswer}
    % your answer here
  \end{subanswer}

  \newpage

  \item \textbf{Intersecting Lines Angle} (10 points)\\
  Intersecting lines $r$, $s$, and $t$ are shown below.
  \insertimage{0.35}{images/2025_06_15_d0312806c0eedc278ae1g-03}{reference attached}
  What is the value of $x$?
  \begin{subanswer}
    % your answer here
  \end{subanswer}

  \item \textbf{Parallel Lines Angle} (10 points)\\
  \insertimage{0.35}{images/2025_06_15_d0312806c0eedc278ae1g-04}{reference attached}
  In the figure, line $m$ is parallel to line $n$. What is the value of $w$?
  \begin{enumerate}[label=(\Alph*)]
    \item 17
    \item 30
    \item 70
    \item 170
  \end{enumerate}
  \begin{subanswer}
    % your answer here
  \end{subanswer}

  \newpage

  \item \textbf{Isosceles Triangle Angle} (10 points)\\
  \insertimage{0.20}{images/cus07.png}{reference attached}
  In the given triangle, $AB=AC$ and $\angle ABC$ has a measure of $67^{\circ}$. What is the value of $x$?
  \begin{enumerate}[label=(\Alph*)]
    \item 36
    \item 46
    \item 58
    \item 70
  \end{enumerate}
  \begin{subanswer}
    % your answer here
  \end{subanswer}

  \item \textbf{Intersecting Segments Angle} (10 points)\\
  \insertimage{0.40}{images/2025_06_15_d0312806c0eedc278ae1g-06}{reference attached}
  
  In the figure above, $\overline{MQ}$ and $\overline{NR}$ intersect at point $P$, $NP=QP$, and $MP=PR$. What is the measure, in degrees, of $\angle QMR$? (Disregard the degree symbol when gridding your answer.)
  \begin{subanswer}
    % your answer here
  \end{subanswer}


  \newpage

  \item \textbf{Similar Triangles Angle} (10 points)\\
  \insertimage{0.35}{images/cus08.png}{reference attached}
  Right triangles $PQR$ and $STU$ are similar, where $P$ corresponds to $S$. If the measure of angle $Q$ is $18^{\circ}$, what is the measure of angle $S$?
  \begin{enumerate}[label=(\Alph*)]
    \item $18^{\circ}$
    \item $72^{\circ}$
    \item $82^{\circ}$
    \item $162^{\circ}$
  \end{enumerate}
  \begin{subanswer}
    % your answer here
  \end{subanswer}

  \item \textbf{Parallel Lines with Transversal} (10 points)\\
  \insertimage{0.35}{images/2025_06_15_d0312806c0eedc278ae1g-08}{reference attached}
  In the figure, line $q$ is parallel to line $r$, and both lines are intersected by line $s$. If $y=2x+8$, what is the value of $x$?
  \begin{subanswer}
    % your answer here
  \end{subanswer}

  \newpage

  \item \textbf{Parallel Lines Angle Relationship} (10 points)\\
  \insertimage{0.35}{images/2025_06_15_d0312806c0eedc278ae1g-09}{reference attached}
  In the figure above, lines $\ell$ and $m$ are parallel, $y=20$, and $z=60$. What is the value of $x$?
  \begin{enumerate}[label=(\Alph*)]
    \item 120
    \item 100
    \item 90
    \item 80
  \end{enumerate}
  \begin{subanswer}
    % your answer here
  \end{subanswer}

  \item \textbf{Parallel Lines Proportionality} (10 points)\\
  \insertimage{0.30}{images/2025_06_15_d0312806c0eedc278ae1g-10}{reference attached}
  In the figure above, $\overline{AF}$, $\overline{BE}$, and $\overline{CD}$ are parallel. Points $B$ and $E$ lie on $\overline{AC}$ and $\overline{FD}$, respectively. If $AB=9$, $BC=18.5$, and $FE=8.5$, what is the length of $\overline{ED}$, to the nearest tenth?
  \begin{enumerate}[label=(\Alph*)]
    \item 16.8
    \item 17.5
    \item 18.4
    \item 19.6
  \end{enumerate}
  \begin{subanswer}
    % your answer here
  \end{subanswer}

  \newpage

  \item \textbf{Similar Triangles Sine Value} (10 points)\\
  Triangle $FGH$ is similar to triangle $JKL$, where angle $F$ corresponds to angle $J$ and angles $G$ and $K$ are right angles. If $\sin(F)=\frac{308}{317}$, what is the value of $\sin(J)$?
  \begin{enumerate}[label=(\Alph*)]
    \item $\frac{75}{317}$
    \item $\frac{308}{317}$
    \item $\frac{317}{308}$
    \item $\frac{317}{75}$
  \end{enumerate}
  \begin{subanswer}
    % your answer here
  \end{subanswer}

  \item \textbf{Right Triangle Trigonometric Relationship} (10 points)\\
  In right triangle $RST$, the sum of the measures of angle $R$ and angle $S$ is 90 degrees. The value of $\sin(R)$ is $\frac{\sqrt{15}}{4}$. What is the value of $\cos(S)$?
  \begin{enumerate}[label=(\Alph*)]
    \item $\frac{\sqrt{15}}{15}$
    \item $\frac{\sqrt{15}}{4}$
    \item $\frac{4\sqrt{15}}{15}$
    \item $\sqrt{15}$
  \end{enumerate}
  \begin{subanswer}
    % your answer here
  \end{subanswer}

  \newpage

  \item \textbf{Tangent Value Calculation} (10 points)\\
  \insertimage{0.30}{images/cus09.png}{reference attached}
  In the figure above, what is the value of $\tan(A)$?
  \begin{enumerate}[label=(\Alph*)]
    \item $\frac{20}{29}$
    \item $\frac{21}{29}$
    \item $\frac{20}{21}$
    \item $\frac{21}{20}$
  \end{enumerate}
  \begin{subanswer}
    % your answer here
  \end{subanswer}

  \item \textbf{Right Triangle Side Length} (10 points)\\
  \insertimage{0.15}{images/cus10.png}{reference attached}
  Triangle $ABC$ above is a right triangle, and $\sin(B)=\frac{5}{13}$. What is the length of side $\overline{BC}$?
  \begin{subanswer}
    % your answer here
  \end{subanswer}

  \newpage

  \item \textbf{Pythagorean Theorem Application} (10 points)\\
  \insertimage{0.30}{images/cus11.png}{reference attached}
  Which equation shows the relationship between the side lengths of the given triangle?
  \begin{enumerate}[label=(\Alph*)]
    \item $4b=19$
    \item $4+b=19$
    \item $4^{2}+b^{2}=19^{2}$
    \item $4^{2}-b^{2}=19^{2}$
  \end{enumerate}
  \begin{subanswer}
    % your answer here
  \end{subanswer}
\end{enumerate}