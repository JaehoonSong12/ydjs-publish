
\section*{ID: f5c3e3b8}
Which expression is equivalent to $\left(m^{4} q^{4} z^{-1}\right)\left(m q^{5} z^{3}\right)$, where $m, q$, and $z$ are positive?\\
A. $m^{4} q^{20} z^{-3}$\\
B. $m^{5} q^{9} z^{2}$\\
C. $m^{6} q^{8} z^{-1}$\\
D. $m^{20} q^{12} z^{-2}$

ID: dd4ab4c4

$$
4 a^{2}+20 a b+25 b^{2}
$$

Which of the following is a factor of the polynomial above?\\
A. $a+b$\\
B. $2 a+5 b$\\
C. $4 a+5 b$\\
D. $4 a+25 b$

\section*{ID: b8caaf84}
If $p=3 x+4$ and $v=x+5$, which of the following is equivalent to $p v-2 p+v$ ?\\
A. $3 x^{2}+12 x+7$\\
B. $3 x^{2}+14 x+17$\\
C. $3 x^{2}+19 x+20$\\
D. $3 x^{2}+26 x+33$

ID: e312081b

$$
(x+5)+(2 x-3)
$$

Which of the following is equivalent to the given expression?\\
A. $3 x-2$\\
B. $3 x+2$\\
C. $3 x-8$\\
D. $3 x+8$

Which expression is equivalent to $\frac{8 x(x-7)-3(x-7)}{2 x-14}$, where $x>7$ ?\\
A. $\frac{x-7}{5}$\\
B. $\frac{8 x-3}{2}$\\
C. $\frac{8 x^{2}-3 x-14}{2 x-14}$\\
D. $\frac{8 x^{2}-3 x-77}{2 x-14}$

\section*{ID: ad2ec615}
Which of the following is equivalent to the expression $x^{4}-x^{2}-6$ ?\\
A. $\left(x^{2}+1\right)\left(x^{2}-6\right)$\\
B. $\left(x^{2}+2\right)\left(x^{2}-3\right)$\\
c. $\left(x^{2}+3\right)\left(x^{2}-2\right)$\\
D. $\left(x^{2}+6\right)\left(x^{2}-1\right)$

\section*{ID: 42c71eb5}
$$
(2 x+5)^{2}-(x-2)+2(x+3)
$$

Which of the following is equivalent to the expression above?\\
A. $4 x^{2}+21 x+33$\\
B. $4 x^{2}+21 x+29$\\
C. $4 x^{2}+x+29$\\
D. $4 x^{2}+x+33$

\section*{ID: 371cbf6b}
$$
(a x+3)\left(5 x^{2}-b x+4\right)=20 x^{3}-9 x^{2}-2 x+12
$$

The equation above is true for all $x$, where $a$ and $b$ are constants. What is the value of $a b$ ?\\
A. 18\\
B. 20\\
C. 24\\
D. 40

\section*{ID: a05bd3a4}
Which of the following expressions is equivalent to $x^{2}-5$ ?\\
A. $(x+\sqrt{5})^{2}$\\
B. $(x-\sqrt{5})^{2}$\\
c. $(x+\sqrt{5})(x-\sqrt{5})$\\
D. $(x+5)(x-1)$

Which of the following expressions is(are) a factor of $3 x^{2}+20 x-63$ ?\\
I. $x-9$\\
II. $3 x-7$\\
A. I only\\
B. II only\\
C. I and II\\
D. Neither I nor II

ID: 40c09d66

If $\frac{\sqrt{x^{5}}}{\sqrt[3]{x^{4}}}=x^{\frac{a}{b}} \quad$ for all positive values of $x$, what is the value of $\frac{a}{b}$ ?

The expression $90 y^{5}-54 y^{4}$ is equivalent to $r y^{4}(15 y-9)$, where $r$ is a constant. What is the value of $r$ ?

$$
\frac{2}{x-2}+\frac{3}{x+5}=\frac{x+t}{(x-2)(x+5)}
$$

The equation above is true for all $x>2$, where $r$ and $t$ are positive constants. What is the value of $r t$ ?\\
A. -20\\
B. 15\\
C. 20\\
D. 60

\section*{ID: cc776a04}
Which of the following is an equivalent form of $(1.5 x-2.4)^{2}-\left(5.2 x^{2}-6.4\right)$ ?\\
A. $-2.2 x^{2}+1.6$\\
B. $-2.2 x^{2}+11.2$\\
C. $-2.95 x^{2}-7.2 x+12.16$\\
D. $-2.95 x^{2}-7.2 x+0.64$

$$
\sqrt[5]{70 n}(\sqrt[6]{70 n})^{2}
$$

For what value of $x$ is the given expression equivalent to $(70 n)^{30 x}$, where $n>1$ ?






































\section*{ID: 463 eec13}
If $x \neq 0$, which of the following expressions is\\
equivalent to $\frac{\sqrt{16 x^{4} y^{8}}}{x^{3}}$ ?\\
A. $8 x^{2} y^{4}$\\
B. $4 x y^{4}$\\
C. $4 x^{-2} y^{2}$\\
D. $4 x^{-1} y^{4}$

$$
0.36 x^{2}+0.63 x+1.17
$$

The given expression can be rewritten as $a\left(4 x^{2}+7 x+13\right)$, where $a$ is a constant. What is the value of $a$ ?

\section*{ID: a1bf1c4e}
$$
x^{2}+6 x+4
$$

Which of the following is equivalent to the expression above?\\
A. $(x+3)^{2}+5$\\
B. $(x+3)^{2}-5$\\
C. $(x-3)^{2}+5$\\
D. $(x-3)^{2}-5$

In the expression $3\left(2 x^{2}+p x+8\right)-16 x(p+4), p$ is a constant. This expression is equivalent to the expression $6 x^{2}-155 x+24$. What is the value of $p$ ?\\
A. -3\\
B. 7\\
C. 13\\
D. 155

\section*{ID: 6d04c89d}
The expression $\frac{24}{6 x+42}$ is equivalent to $\frac{4}{x+b}$, where $b$ is a constant and $x>0$. What is the value of $b$ ?\\
A. 7\\
B. 10\\
C. 24\\
D. 252

Which expression is equivalent to $\left(2 x^{2}-4\right)-\left(-3 x^{2}+2 x-7\right)$ ?\\
A. $5 x^{2}-2 x+3$\\
B. $5 x^{2}+2 x-3$\\
C. $-x^{2}-2 x-11$\\
D. $-x^{2}+2 x-11$

The expression $\frac{x^{-2} y^{\frac{1}{2}}}{x^{\frac{1}{3}} y^{-1}}$, where $x>1$ and $y>1$, is equivalent to which of the following?\\
A. $\frac{\sqrt{y}}{\sqrt[3]{x^{2}}}$\\
B. $\frac{y \sqrt{y}}{\sqrt[3]{x^{2}}}$\\
C. $\frac{y \sqrt{y}}{x \sqrt{x}}$\\
D. $\frac{y \sqrt{y}}{x^{2} \sqrt[3]{x}}$

\section*{ID: ffdbcad4}
The expression $4 x^{2}+b x-45$, where $b$ is a constant, can be rewritten as $(h x+k)(x+j)$, where $h, k$, and $j$ are integer constants. Which of the following must be an integer?\\
A. $\frac{b}{h}$\\
B. $\frac{b}{k}$\\
C. $\frac{45}{h}$\\
D. $\frac{45}{k}$

\section*{ID: 4a5af623}
Which expression is a factor of $2 x^{2}+38 x+10$ ?\\
A. 2\\
B. $5 x$\\
C. $38 x$\\
D. $2 x^{2}$

\section*{ID: 22fd3e1f}
$$
\begin{gathered}
f(x)=x^{3}-9 x \\
g(x)=x^{2}-2 x-3
\end{gathered}
$$

Which of the following expressions is\\
equivalent to $\frac{f(x)}{g(x)}$, for $x>3$ ?\\
A. $\frac{1}{x+1}$\\
B. $\frac{x+3}{x+1}$\\
c. $\frac{x(x-3)}{x+1}$\\
D. $\frac{x(x+3)}{x+1}$

The expression $\frac{1}{3} x^{2}-2$ can be rewritten as $\frac{1}{3}(x-k)(x+k)$, where $k$ is a positive constant. What is the value of $k$ ?\\
A. 2\\
B. 6\\
C. $\sqrt{2}$\\
D. $\sqrt{6}$

\section*{ID: 49efde89}
The expression $2 x^{2}+a x$ is equivalent to $x(2 x+7)$ for some constant $a$. What is the value of $a$ ?\\
A. 2\\
B. 3\\
C. 4\\
D. 7

Which expression is equivalent to $\left(7 x^{3}+7 x\right)-\left(6 x^{3}-3 x\right)$ ?\\
A. $x^{3}+10 x$\\
B. $-13 x^{3}+10 x$\\
C. $-13 x^{3}+4 x$\\
D. $x^{3}+4 x$

\section*{ID: 26eb61c1}
Which expression is equivalent to $6 x^{8} y^{2}+12 x^{2} y^{2}$ ?\\
A. $6 x^{2} y^{2}\left(2 x^{6}\right)$\\
B. $6 x^{2} y^{2}\left(x^{4}\right)$\\
C. $6 x^{2} y^{2}\left(x^{6}+2\right)$\\
D. $6 x^{2} y^{2}\left(x^{4}+2\right)$

\section*{ID: 9ed9f54d}
Which of the following is equivalent to $2\left(x^{2}-x\right)+3\left(x^{2}-x\right)$ ?\\
A. $5 x^{2}-5 x$\\
B. $5 x^{2}+5 x$\\
C. $5 x$\\
D. $5 x^{2}$

\section*{ID: 42f19012}
Which expression is equivalent to $a^{\frac{11}{12}}$, where $a>0$ ?\\
A. $\sqrt[12]{a^{132}}$\\
B. $\sqrt[144]{a^{132}}$\\
C. $\sqrt[121]{a^{132}}$\\
D. $\sqrt[11]{a^{132}}$

Which of the following is equivalent to $2 x^{3}+4$ ?\\
A. $4\left(x^{3}+4\right)$\\
B. $4\left(x^{3}+2\right)$\\
C. $2\left(x^{3}+4\right)$\\
D. $2\left(x^{3}+2\right)$

The sum of $-2 x^{2}+x+31$ and $3 x^{2}+7 x-8$ can be written in the form $a x^{2}+b x+c$ , where $a, b$, and $c$ are constants. What is the value of $a+b+c$ ?

ID: a391ed22\\
$\left(\frac{1}{2} x+\frac{3}{2}\right)\left(\frac{3}{2} x+\frac{1}{2}\right)$\\
The expression above is equivalent to $a x^{2}+b x+c$, where $a, b$, and $c$ are constants. What is the value of $b$ ?

\section*{ID: beb86a0c}
Which expression is equivalent to $9 x^{2}+5 x$ ?\\
A. $x(9 x+5)$\\
B. $5 x(9 x+1)$\\
C. $9 x(x+5)$\\
D. $x^{2}(9 x+5)$

\section*{ID: 6e06a0a7}
Which of the following expressions is equivalent to $2 a^{2}(a+3)$ ?\\
A. $5 a^{3}$\\
B. $8 a^{5}$\\
c. $2 a^{3}+3$\\
D. $2 a^{3}+6 a^{2}$

Which of the following is\\
equivalent to $\left(a+\frac{b}{2}\right)^{2}$ ?\\
A. $a^{2}+\frac{b^{2}}{2}$\\
B. $a^{2}+\frac{b^{2}}{4}$\\
c. $a^{2}+\frac{a b}{2}+\frac{b^{2}}{2}$\\
D. $a^{2}+a b+\frac{b^{2}}{4}$

Which expression is equivalent to $5 x^{2}-50 x y^{2}$ ?\\
A. $5 x\left(x-10 y^{2}\right)$\\
B. $5 x\left(x-50 y^{2}\right)$\\
C. $5 x^{2}\left(10 x y^{2}\right)$\\
D. $5 x^{2}\left(50 x y^{2}\right)$

Which expression is equivalent to $\frac{y+12}{x-8}+\frac{y(x-8)}{x^{2} y-8 x y}$ ?\\
A. $\frac{x y+y+4}{x^{3} y-16 x^{2} y+64 x y}$\\
B. $\frac{x y+9 y+12}{x^{2} y-8 x y+x-8}$\\
C. $\frac{x y^{2}+13 x y-8 y}{x^{2} y-8 x y}$\\
D. $\frac{x y^{2}+13 x y-8 y}{x^{3} y-16 x^{2} y+64 x y}$

One of the factors of $2 x^{3}+42 x^{2}+208 x$ is $x+b$, where $b$ is a positive constant. What is the smallest possible value of $b$ ?

\section*{ID: 67e866b5}
Which expression is equivalent to $9 x^{2}+7 x^{2}+9 x$ ?\\
A. $63 x^{4}+9 x$\\
B. $9 x^{2}+16 x$\\
C. $25 x^{5}$\\
D. $16 x^{2}+9 x$

\section*{ID: fd65f47f}
Which expression is equivalent to $\left(2 x^{2}+x-9\right)+\left(x^{2}+6 x+1\right)$ ?\\
A. $2 x^{2}+7 x+10$\\
B. $2 x^{2}+6 x-8$\\
C. $3 x^{2}+7 x-10$\\
D. $3 x^{2}+7 x-8$

\section*{ID: 12e7faf8}
The equation $\frac{x^{2}+6 x-7}{x+7}=a x+d$ is true for all $x \neq-7$, where $a$ and $d$ are integers. What is the value of $a+d$ ?\\
A. -6\\
B. -1\\
C. 0\\
D. 1

Which of the following expressions is\\
equivalent to $\frac{x^{2}-2 x-5}{x-3}$ ?\\
A. $x-5-\frac{20}{x-3}$\\
B. $x-5-\frac{10}{x-3}$\\
c. $x+1-\frac{8}{x-3}$\\
D. $x+1-\frac{2}{x-3}$

\section*{ID: c3a72da5}
Which of the following is equivalent to the sum of $3 x^{4}+2 x^{3}$ and $4 x^{4}+7 x^{3}$ ?\\
A. $16 x^{14}$\\
B. $7 x^{8}+9 x^{6}$\\
C. $12 x^{4}+14 x^{3}$\\
D. $7 x^{4}+9 x^{3}$

$$
\left(7532+100 y^{2}\right)+10\left(10 y^{2}-110\right)
$$

The expression above can be written in the form $a y^{2}+b$, where $a$ and $b$ are constants. What is the value of $a+b$ ?

The expression $6 \sqrt[5]{3^{5} x^{45}} \cdot \sqrt[8]{2^{8} x}$ is equivalent to $a x^{b}$, where $a$ and $b$ are positive constants and $x>1$. What is the value of $a+b$ ?

ID: 16de54c7\\
$2 x^{2}+5 x-12$

If the given expression is rewritten in the form $(2 x-3)(x+k)$, where $k$ is a constant, what is the value of $k$ ?

Which expression represents the product of $\left(x^{-6} y^{3} z^{5}\right)$ and $\left(x^{4} z^{5}+y^{8} z^{-7}\right)$ ?\\
A. $x^{-2} z^{10}+y^{11} z^{-2}$\\
B. $x^{-2} z^{10}+x^{-6} z^{-2}$\\
C. $x^{-2} y^{3} z^{10}+y^{8} z^{-7}$\\
D. $x^{-2} y^{3} z^{10}+x^{-6} y^{11} z^{-2}$

\section*{ID: f89e1d6f}
If $a=c+d$, which of the following is equivalent to the expression $x^{2}-c^{2}-2 c d-d^{2}$ ?\\
A. $(x+a)^{2}$\\
B. $(x-a)^{2}$\\
C. $(x+a)(x-a)$\\
D. $x^{2}-a x-a^{2}$

$$
\left(2 x^{3}+3 x\right)\left(x^{3}-2 x\right)
$$

Which of the following is equivalent to the expression above?\\
A. $x^{3}+5 x$\\
B. $3 x^{3}+x$\\
c. $2 x^{6}-x^{4}-6 x^{2}$\\
D. $3 x^{6}-x^{4}-6 x^{2}$

\section*{ID: 3e9cc0c2}
Which of the following is equivalent to\\
$(1-p)\left(1+p+p^{2}+p^{3}+p^{4}+p^{5}+p^{6}\right)$ ?\\
A. $1-p^{8}$\\
B. $1-p^{7}$\\
c. $1-p^{6}$\\
D. $1-p^{5}$

ID: 7348f046\\
$(2 x+3)-(x-7)$

Which of the following is equivalent to the given expression?\\
A. $x-4$\\
B. $3 x-4$\\
C. $x+10$\\
D. $2 x^{2}+21$

$$
\left(\frac{1}{2} x+3\right)-\left(\frac{2}{3} x-5\right)
$$

Which of the following is equivalent to the expression above?\\
A. $-\frac{1}{6} x+8$\\
B. $-\frac{1}{6} x-2$\\
c. $-\frac{1}{3} x^{2}+\frac{1}{2} x+15$\\
D. $-\frac{1}{3} x^{2}-\frac{9}{2} x-15$

ID: 8838a672\\
$\left(4 x^{3}-5 x^{2}+3\right)-\left(6 x^{3}+2 x^{2}-x\right)$

Which of the following expressions is equivalent to the expression above?\\
A. $-10 x^{3}-3 x^{2}+x+3$\\
B. $-2 x^{3}-7 x^{2}+x+3$\\
C. $-2 x^{3}-3 x^{2}+x+3$\\
D. $10 x^{3}-7 x^{2}-x+3$

\section*{ID: 0b3d25c5}
Which of the following is equivalent to $\sqrt[4]{x^{2}+8 x+16}$, where $x>0$ ?\\
A. $(x+4)^{4}$\\
B. $(x+4)^{2}$\\
C. $(x+4)$\\
$(x+4)^{\frac{1}{2}}$\\
D.

If $a$ and $c$ are positive numbers, which of the following is equivalent to $\sqrt{(a+c)^{3}} \cdot \sqrt{a+c}$ ?\\
A. $a+c$\\
B. $a^{2}+c^{2}$\\
C. $a^{2}+2 a c+c^{2}$\\
D. $a^{2} c^{2}$

Which expression is equivalent to $8+d^{2}+3$ ?\\
A. $d^{2}+24$\\
B. $d^{2}+11$\\
C. $d^{2}+5$\\
D. $d^{2}-11$

\section*{ID: fb96a5b3}
Which of the following expressions is equivalent to $2(a b-3)+2$ ?\\
A. $2 a b-1$\\
B. $2 a b-4$\\
C. $2 a b-5$\\
D. $2 a b-8$

$$
\left(5 x^{3}-3\right)-\left(-4 x^{3}+8\right)
$$

The given expression is equivalent to $b x^{3}-11$, where $b$ is a constant. What is the value of $b$ ?

\section*{ID: e597050f}
Which expression is equivalent to $9 x+6 x+2 y+3 y$ ?\\
A. $3 x+5 y$\\
B. $6 x+8 y$\\
C. $12 x+8 y$\\
D. $15 x+5 y$

\section*{ID: 1e8d7183}
Which expression is equivalent to $256 w^{2}-676$ ?\\
A. $(16 w-26)(16 w-26)$\\
B. $(8 w-13)(8 w+13)$\\
C. $(8 w-13)(8 w-13)$\\
D. $(16 w-26)(16 w+26)$

\section*{ID: 0354c7de}
$5 x+15$

Which of the following is equivalent to the given expression?\\
A. $5(x+3)$\\
B. $5(x+10)$\\
C. $5(x+15)$\\
D. $5(x+20)$

\section*{ID: 4eaf0a3a}
Which expression is equivalent to $\sqrt[7]{x^{9} y^{9}}$, where $x$ and $y$ are positive?\\
A. msup\\
B. msup\\
C. msup\\
D. msup

ID: c602140f\\
$(x-11 y)(2 x-3 y)-12 y(-2 x+3 y)$

Which of the following is equivalent to the expression above?\\
A. $x-23 y$\\
B. $2 x^{2}-x y-3 y^{2}$\\
C. $2 x^{2}+24 x y+36 y^{2}$\\
D. $2 x^{2}-49 x y+69 y^{2}$

\section*{ID: fd4b2aa0}
Which expression is equivalent to $12 x^{3}-5 x^{3}$ ?\\
A. $7 x^{6}$\\
B. $17 x^{3}$\\
C. $7 x^{3}$\\
D. $17 x^{6}$

\section*{ID: 5d93c782}
Which expression is equivalent to $x^{2}+3 x-40$ ?\\
A. $(x-4)(x+10)$\\
B. $(x-5)(x+8)$\\
C. $(x-8)(x+5)$\\
D. $(x-10)(x+4)$

Which of the following expressions is equivalent to the sum of $\left(r^{3}+5 r^{2}+7\right)$ and $\left(r^{2}+8 r+12\right) ?$\\
A. $r^{5}+13 r^{3}+19$\\
B. $2 r^{3}+13 r^{2}+19$\\
C. $r^{3}+5 r^{2}+7 r+12$\\
D. $r^{3}+6 r^{2}+8 r+19$

\section*{ID: d4d513ff}
Which expression is equivalent to $12 x+27$ ?\\
A. $12(9 x+1)$\\
B. $27(12 x+1)$\\
C. $3(4 x+9)$\\
D. $3(9 x+24)$


