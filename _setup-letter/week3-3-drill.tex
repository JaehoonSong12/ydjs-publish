\newpage\handout
{Drill Problems: Week 03-3}
{\textsc{Scholastic Aptitude Test (SAT)}}
{\href{https://creativecommons.org/licenses/by-nc-sa/4.0/}{CC BY-NC-SA 4.0 license}}
{Author: \BookAuthor}{Release: \generatedOn}
{SAT: Drill Problems (3-3)}
% Copyright & Disclaimer
\newcommand{\BookAuthor}{Jaehoon Song (Lecturer)}

\begin{center}
  \begin{minipage}{0.85\textwidth}
    {\small\textbf{Purpose and Usage:}}\\[0.2cm]
    {\footnotesize
    This material has been developed for internal training and educational 
    purposes at Hans edu LLC. It is intended for use within our organization 
    and should not be distributed, sold, or used for commercial purposes 
    outside of our educational programs.}\\[0.5cm]
    
    {\small\textbf{For Our Community:}}\\[0.2cm]
    {\footnotesize
    Students and staff are welcome to use this material in their studies and 
    teaching at Hans edu LLC. While we encourage active engagement with the 
    content, please respect that this is proprietary material. Any 
    reproduction or distribution outside of our organization's educational 
    activities is not permitted.}\\[0.5cm]
    
    {\small\textbf{Content and Attribution:}}\\[0.2cm]
    {\footnotesize
    This material represents our adaptation of various established mathematics 
    textbooks, reorganized and enhanced for our teaching context. While we've 
    added our own pedagogical improvements, we maintain proper attribution to 
    original sources. This work is shared under the Creative Commons 
    Attribution-NonCommercial-ShareAlike (CC BY-NC-SA) license, allowing 
    internal use and adaptation while respecting the original creators' rights.\\[0.2cm]
    \small Key Permissions under CC BY-NC-SA 4.0:
    \begin{itemize}
      \item credit the creator.
      \item no commercial use.
      \item new creations must carry the same license.
      \item copy and redistribute the material in any medium or format
      \item remix, transform, and build upon the material
    \end{itemize}}
    % Some content may be derived from sources under the CC0 1.0 Universal 
    % license, which allows for free use, modification, and distribution.}\\[0.5cm]
    {\small\textbf{Quality Assurance:}}\\[0.2cm]
    {\footnotesize
    We have carefully reviewed this material for accuracy and clarity. However, 
    as with any educational resource, we encourage critical engagement and 
    verification of concepts. If you notice any issues or have suggestions for 
    improvement, please bring them to our attention.}
  \end{minipage}
  \vspace{2cm} \\
  {\small © \the\year\ Hans edu LLC. All rights reserved.}\\[0.5cm]
  {\small Written by \BookAuthor}\\[1cm]
\end{center}

\newpage


\begin{enumerate}
  \item \textbf{Circle Equations} (10 points)\\
  Circle A in the $xy$-plane has the equation $(x+5)^2+(y-5)^2=4$. Circle B has the same center as circle A. The radius of circle B is two times the radius of circle A. The equation defining circle B in the $xy$-plane is $(x+5)^2+(y-5)^2=k$, where $k$ is a constant. What is the value of $k$?
  \begin{subanswer}
    % your answer here
  \end{subanswer}

  \item \textbf{Circle Diameter} (10 points)\\
  What is the diameter of the circle in the $xy$-plane with equation $(x-5)^2+(y-3)^2=16$?\\
  \begin{enumerate}[label=(\Alph*)]
    \item 4
    \item 8
    \item 16
    \item 32
  \end{enumerate}
  \begin{subanswer}
    % your answer here
  \end{subanswer}

  \item \textbf{Arc and Angle Measure} (10 points)\\
  Point $O$ is the center of a circle. The measure (central angle) of arc $RS$ on this 
  circle is $100^{\circ}$. What is the measure, in degrees, of its associated angle (major-minor relationship) $ROS$?
  \begin{subanswer}
    % your answer here
  \end{subanswer}

  \item \textbf{Circle Radius} (10 points)\\
  The equation $(x+6)^2+(y+3)^2=121$ defines a circle in the $xy$-plane. What is the radius of the circle?
  \begin{subanswer}
    % your answer here
  \end{subanswer}

  \newpage

  \item \textbf{Tangent Line Slope} (10 points)\\
  A circle in the $xy$-plane has its center at $(-4,-6)$. Line $k$ is tangent to this circle at the point $(-7,-7)$. What is the slope of line $k$?\\
  \begin{enumerate}[label=(\Alph*)]
    \item $-3$
    \item $-\frac{1}{3}$
    \item $\frac{1}{3}$
    \item $3$
  \end{enumerate}
  \begin{subanswer}
    % your answer here
  \end{subanswer}

  \item \textbf{Triangle Area} (10 points)\\
  In triangle $A B C$, the measure of angle $B$ is $90^{\circ}$ and $\overline{B D}$ is an altitude of the triangle. The length of $\overline{A B}$ is 15 and the length of $\overline{A C}$ is 23 greater than the length of $\overline{A B}$. What is the value of $\frac{B C}{B D}$?\\
  \begin{enumerate}[label=(\Alph*)]
    \item $\frac{15}{38}$
    \item $\frac{15}{23}$
    \item $\frac{23}{15}$
    \item $\frac{38}{15}$
  \end{enumerate}
  \begin{subanswer}
    % your answer here
  \end{subanswer}

  \item \textbf{Triangle Angle} (10 points)\\
  In $\triangle X Y Z$, the measure of $\angle X$ is $24^{\circ}$ and the measure of $\angle Y$ is $98^{\circ}$. What is the measure of $\angle Z$?\\
  \begin{enumerate}[label=(\Alph*)]
    \item $58^{\circ}$
    \item $74^{\circ}$
    \item $122^{\circ}$
    \item $212^{\circ}$
  \end{enumerate}
  \begin{subanswer}
    % your answer here
  \end{subanswer}

  \newpage

  \item \textbf{Tree Height} (10 points)\\
  Two nearby trees are perpendicular to the ground, which is flat. One of these trees is 10 feet tall and has a shadow that is 5 feet long. At the same time, the shadow of the other tree is 2 feet long. How tall, in feet, is the other tree?\\
  \begin{enumerate}[label=(\Alph*)]
    \item 3
    \item 4
    \item 8
    \item 27
  \end{enumerate}
  \begin{subanswer}
    % your answer here
  \end{subanswer}

  \item \textbf{Parallel Lines} (10 points)\\
  \insertimage{0.30}{images/2025_06_15_d0312806c0eedc278ae1g-14}{reference attached}
  In the figure, line $m$ is parallel to line $n$. What is the value of $w$?\\
  \begin{enumerate}[label=(\Alph*)]
    \item 13
    \item 34
    \item 66
    \item 134
  \end{enumerate}
  \begin{subanswer}
    % your answer here
  \end{subanswer}

  \newpage

  \item \textbf{Triangle Congruence} (10 points)\\
  In triangles $A B C$ and $D E F$, angles $B$ and $E$ each have measure $27^{\circ}$ and angles $C$ and $F$ each have measure $41^{\circ}$. Which additional piece of information is sufficient to determine whether triangle $A B C$ is congruent to triangle $D E F$?\\
  \begin{enumerate}[label=(\Alph*)]
    \item The measure of angle $A$
    \item The length of side $A B$
    \item The lengths of sides $B C$ and $E F$
    \item No additional information is necessary
  \end{enumerate}
  \begin{subanswer}
    % your answer here
  \end{subanswer}

  \item \textbf{Triangle Similarity} (10 points)\\
  In triangles $L M N$ and $R S T$, angles $L$ and $R$ each have measure $60^{\circ}, L N=10$, and $R T=30$. Which additional piece of information is sufficient to prove that triangle $L M N$ is similar to triangle $R S T$?\\
  \begin{enumerate}[label=(\Alph*)]
    \item $M N=7$ and $S T=7$
    \item $M N=7$ and $S T=21$
    \item The measures of angles $M$ and $S$ are $70^{\circ}$ and $60^{\circ}$, respectively.
    \item The measures of angles $M$ and $T$ are $70^{\circ}$ and $50^{\circ}$, respectively.
  \end{enumerate}
  \begin{subanswer}
    % your answer here
  \end{subanswer}

  \newpage

  \item \textbf{Triangle Ratio} (10 points)\\
  \insertimage{0.30}{images/cus01.png}{reference attached}
  Triangles $A B C$ and $D E F$ are shown above. Which of the following is equal to the ratio $\frac{B C}{A B}$?\\
  \begin{enumerate}[label=(\Alph*)]
    \item $\frac{D E}{D F}$
    \item $\frac{D F}{D E}$
    \item $\frac{D F}{E F}$
    \item $\frac{E F}{D E}$
  \end{enumerate}
  \begin{subanswer}
    % your answer here
  \end{subanswer}

  \item \textbf{Circle Length} (10 points)\\
  \insertimage{0.30}{images/2025_06_15_d0312806c0eedc278ae1g-18}{reference attached}

  In the figure above, what is the length of $\overline{N Q}$?\\
  \begin{enumerate}[label=(\Alph*)]
    \item 2.2
    \item 2.3
    \item 2.4
    \item 2.5
  \end{enumerate}
  \begin{subanswer}
    % your answer here
  \end{subanswer}

  \item \textbf{Triangle Angle} (10 points)\\
  Triangle $X Y Z$ is similar to triangle $R S T$ such that $X, Y$, and $Z$ correspond to $R, S$, and $T$, respectively. The measure of $\angle Z$ is $20^{\circ}$ and $2 X Y=R S$. What is the measure of $\angle T$?\\
  \begin{enumerate}[label=(\Alph*)]
    \item $2^{\circ}$
    \item $10^{\circ}$
    \item $20^{\circ}$
    \item $40^{\circ}$
  \end{enumerate}
  \begin{subanswer}
    % your answer here
  \end{subanswer}

  \item \textbf{Parallel Lines} (10 points)\\
  \insertimage{0.30}{images/2025_06_15_d0312806c0eedc278ae1g-20}{reference attached}

  Note: Figure not drawn to scale.\\
  In the figure shown, lines $r$ and $s$ are parallel, and line $m$ intersects both lines. If $y<65$, which of the following must be true?\\
  \begin{enumerate}[label=(\Alph*)]
    \item $x<115$
    \item $x>115$
    \item $x+y<180$
    \item $x+y>180$
  \end{enumerate}
  \begin{subanswer}
    % your answer here
  \end{subanswer}
  \newpage

  \item \textbf{Trigonometry} (10 points)\\
  \insertimage{0.30}{images/cus02.png}{reference attached}
  In the triangle shown, what is the value of $\sin x^{\circ}$?
  \begin{subanswer}
    % your answer here
  \end{subanswer}

  \item \textbf{Logo Area} (10 points)\\
  \insertimage{0.30}{images/cus03.png}{reference attached}
  A graphic designer is creating a logo for a company. The logo is shown in the figure above. The logo is in the shape of a trapezoid and consists of three congruent equilateral triangles. If the perimeter of the logo is 20 centimeters, what is the combined area of the shaded regions, in square centimeters, of the logo?\\
  \begin{enumerate}[label=(\Alph*)]
    \item $2 \sqrt{3}$
    \item $4 \sqrt{3}$
    \item $8 \sqrt{3}$
    \item 16
  \end{enumerate}
  \begin{subanswer}
    % your answer here
  \end{subanswer}


  \newpage

  \item \textbf{Triangle Tangent} (10 points)\\
  \insertimage{0.30}{images/cus04.png}{reference attached}
  In the triangle shown, what is the value of $\tan x^{\circ}$?\\
  \begin{enumerate}[label=(\Alph*)]
    \item $\frac{1}{26}$
    \item $\frac{19}{26}$
    \item $\frac{26}{7}$
    \item $\frac{33}{7}$
  \end{enumerate}
  \begin{subanswer}
    % your answer here
  \end{subanswer}

  \item \textbf{Triangle Height} (10 points)\\
  The perimeter of an equilateral triangle is 624 centimeters. The height of this triangle is $k \sqrt{3}$ centimeters, where $k$ is a constant. What is the value of $k$?
  \begin{subanswer}
    % your answer here
  \end{subanswer}

  \newpage

  \item \textbf{Triangle Length} (10 points)\\
  \insertimage{0.30}{images/2025_06_15_f612918e7cf8d6fcced3g-10}{reference attached}
  In the figure above, $\tan B=\frac{3}{4}$. If $B C=15$ and $D A=4$, what is the length of $\overline{D E}$?
  \begin{subanswer}
    % your answer here
  \end{subanswer}
\end{enumerate}



